\chapter{Future Work and Conclusion}
In this dissertation, I have described three projects that explore the design,  development and evaluation of mobile phone food diaries. The BALANCE project incorporated a physical activity sensing unit and energy balance visualization with the food diary. The design process depended on the use of focus groups. An in-lab study characterized the benefits and drawbacks of using a mobile phone food diary based on food indexes to self-monitor dietary intake. Finally, the POND tool and study investigated the use of a food index-based food diary on a mobile phone. 

[JAK: I would definitely like to see some sort of synthesis of all your projects and some overall discussion of what can be learned from each. What were the major findings? What were the lessons you learned? What are the next steps and where do you see the field going? What can others take away from your work? Restate your contributions and summarize how your studies answered them.] 

The BALANCE study taught us that many people find the use of a nutrient-based or database-dependent approach for self-monitoring dietary intake is challenging by nature. Even when the technology is improved (more memory, increased processing power, more natural and easy to use interface), the nature of the detailed approach is too challenging for some people. Providing more benefits in terms of automatically monitoring activity and improved visualization or analysis does not provide enough benefit for most general population. This is particularly true when time of entry is a concern. 

The multi-phase, iterative design of the BALANCE project also provided insight into the evaluation of \textit{in situ} food diary use. The focus groups and standardized questionnaires provided feedback about the current iteration of the BALANCE software, but the lack of more detailed metrics prevented us from reconciling the subjective feedback from software performance and features. The more detailed usage metrics from the validation study helped us to discover that BALANCE usage was similar to what other researchers reported. 

A detailed analysis of the usage metrics allowed us to better understand which metrics may be useful to capture in future work. 

However, we also discovered that researchers have not been consistent with how they report food diary use. 

The in-lab study taught us that using a food index-based approach for informing the design of a mobile phone food diary produced a tool that integrated data entry and analysis, resulting in less time to input. We found that the self-monitoring process can be simplified too much: just because a self-monitoring approach is fast and easy to count correctly does not make it useful (FBQI). This study also taught us that people have nutrition or dietary goals with varying characteristics: some people are more concerned with quality, while others are concerned about quantity. This could impact their willingness to continue using a particular food diary for an extended period of time. 

The in-lab study also provided an in-lab study procedure and valid tasks. 

The POND project consisted of designing and developing a mobile phone food diary based on both behavior change theory and user studies. I evaluated the POND app by performing both in-lab and \textit{in situ} studies. The initial analysis I presented as part of this dissertation included identifying how participants chose to make entries, what entries they were making, and when they were making entries. I found that some participants preferred the overview and quick entry of the +1 buttons; others preferred the accuracy of looking food up in the database; and some participants combined the two approaches. Participant entries consisted primarily of the food group components. The number nutrient component entries was smaller than expected. Finally, participants expressed frustration when they could not retroactively create food entries. The follow-up interviews indicate that participants want a tool that is flexible in regards to the user's schedule. Sometimes their schedule allows them to focus on the food diary right after eating, while other times the user needs to go back in time. 

The POND study proved to us that users have different goals, motivations and resources. Different participants chose different ways to monitor their intake. 

The experience of all three of these projects make one thing clear: there is a disconnect between what users want and are able to do to self-monitor their dietary intake, and what researchers want or need. As researchers, we can provide an rigorous academic justification for why people should keep a timely, detailed journal of what they eat. However, if we are designing tools to change behavior, we need to design tools that accommodate people who are only human. 

Researchers sometimes need highly accurate, very detailed information about what people eat. These areas of research were covered in Chapter \ref{cha:relatedWork}. FINISH this paragraph: don't assume that what is appropriate for researchers who need accuracy is what is appropriate for users. 

Something I hope other researchers can take from this work is an awareness of assumptions that are made about the design and use of food diaries. This area of research has been avoided in the technology community because of its complexity. As we saw with the BALANCE study, building a traditional food diary is not a trivial project. 
