\chapter{Future Work and Conclusion}
In this dissertation, I have described three projects that explore the design,  development and evaluation of mobile phone food diaries. The BALANCE project incorporated a physical activity sensing unit and energy balance visualization with the food diary. The design process depended on the use of focus groups. An in-lab study characterized the benefits and drawbacks of using a mobile phone food diary based on food indexes to self-monitor dietary intake. Finally, the POND tool and study investigated the use of a food index-based food diary on a mobile phone. 

[JAK: I would definitely like to see some sort of synthesis of all your projects and some overall discussion of what can be learned from each. What were the major findings? What were the lessons you learned? What are the next steps and where do you see the field going? What can others take away from your work? Restate your contributions and summarize how your studies answered them.] 

The BALANCE study taught us that many people find the use of a nutrient-based or database-dependent approach for self-monitoring dietary intake is challenging by nature. Even when the technology is improved (more memory, increased processing power, more natural and easy to use interface), the nature of the detailed approach is too challenging for some people. Providing more benefits in terms of automatically monitoring activity and improved visualization or analysis does not provide enough benefit for most general population. This is particularly true when time of entry is a concern. 

The in-lab study taught us that using a food index-based approach for informing the design of a mobile phone food diary produced a tool that integrated data entry and analysis, resulting in less time to input. We found that the self-monitoring process can be simplified too much: just because a self-monitoring approach is fast and easy to count correctly does not make it useful (FBQI). This study also taught us that people have nutrition or dietary goals with varying characteristics: some people are more concerned with quality, while others are concerned about quantity. This could impact their willingness to continue using a particular food diary for an extended period of time. 

The POND studies proved to us that users have different goals, motivations and resources. Different participants chose different ways to monitor their intake. 

Something I hope other researchers can take from this work is an awareness of assumptions that are made about the design and use of food diaries. This area of research has been avoided in the technology community because of its complexity. As we saw with the BALANCE study, building a traditional food diary is not a trivial project. 

\section{Overarching Themes}
\begin{itemize*}
\item choosing participant populations
\item In the POND study, I identified and characterized three different groups of participants. This reflects the distinction between what is being built and what people are willing/wanting to do. 
\item a need for HCI in nutrition studies
\item a need for designers 
\item I haven't seen any work published that focuses on evaluating how different features of an electronic self-monitoring tool impact the ability of an individual to self-monitor over time. 
\end{itemize*}


\begin{itemize*}
\item Disconnect between what researchers need (data to collect-- nutritionally), what users need, and the design process. 
\item Disconnect between what is built and user motivation, goals and resources/constraints 
\item Databases are overwhelming. While on the one hand it's really cool to think that we can have the detailed nutrition information for ANY FOOD at one's fingerprints, it's in general pretty overwhelming. One area of future research should be how to discover the few foods people eat frequently, somewhat frequently, occasionally, and never. 
\item Process for organizing food according to how consumers think about it rather than experts
\item I wanted to investigate how to capture some information without the database. But, reviewing the entire series of work, I've come to accept that IF the database can be perfect, and IF it's easy to find that perfect food, many people will be very very very happy. However, if it's not, they'll be very very very unhappy.  
\item Characterizing the users, their goals, their level of motivation, their level of commitment, and something about the schedule/routine. Do they want all the details? Or will less detail with "frequent touches" be enough? 
\item Investigation characterizing user's nutrition goals
\item How people think about food. That is, given a plate of food, what terms do they first generate to try to find it in the database? Do they look for a combo? Do they identify a manufacturer? Do they want to browse? How does this vary from person to person, and what characteristics or background inform what choice people make? 
\item I see a study: simple app. For a few days, ask people to take a picture of their meal (before they eat), and enter the first word they would search for in a database. 
\item Tools for researchers doing food diary research-a food diary system that's open source, accessible, modular, extensible, so people can get started doing research, we can get people exploring new ideas fast, rather than either requiring them to build something from scratch (problematic) or rely on a closed, commercial system. 
\item HOW to do food diary research
\item A Database-heavy could food diary could encourage people to eat more processed foods
\item Databases give a false sense of accuracy

\end{itemize*}
