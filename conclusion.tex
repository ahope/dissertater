\chapter{Conclusion}
\label{cha:futureWork}
In this dissertation, I described three projects that explore the design,  development and evaluation of mobile phone food diaries. The BALANCE project incorporated a physical activity sensing unit and energy balance visualization with the food diary. The design process depended on the use of focus groups. An in-lab study characterized the benefits and drawbacks of using a mobile phone food diary based on food indexes to self-monitor dietary intake. Finally, the POND tool and study investigated the use of a food index-based food diary on a mobile phone. 

The BALANCE study taught us that many people find the use of a nutrient-based or database-dependent approach for self-monitoring dietary intake is challenging by nature. Even when the technology is improved (more memory, increased processing power, more natural and easy to use interface), the nature of the detailed approach is too challenging for some people. Providing more benefits in terms of automatically monitoring activity and improved visualization or analysis does not provide enough benefit for most general population. This is particularly true when time of entry is a concern. 

The multi-phase, iterative design of the BALANCE project also provided insight into the evaluation of \textit{in situ} food diary use. The focus groups and standardized questionnaires provided feedback about the current iteration of the BALANCE software, but the lack of more detailed metrics prevented us from reconciling the subjective feedback from software performance and features. The more detailed usage metrics from the validation study helped us to discover that BALANCE usage was similar to what other researchers reported. 




The in-lab study taught us that using a food index-based approach for informing the design of a mobile phone food diary produced a tool that integrated data entry and analysis, resulting in less time to input. We found that the self-monitoring process can be simplified too much: just because a self-monitoring approach is fast and easy to count correctly does not make it useful (FBQI). This study also taught us that people have nutrition or dietary goals with varying characteristics: some people are more concerned with quality, while others are concerned about quantity. This could impact their willingness to continue using a particular food diary for an extended period of time. 


The POND project consisted of designing and developing a mobile phone food diary based on both behavior change theory and user studies. I evaluated the POND software by performing both in-lab and \textit{in situ} studies. The preliminary analysis I presented as part of this dissertation included identifying how participants chose to make entries, what entries they were making, and when they were making entries. I found that some participants preferred the overview and quick entry of the +1 buttons; others preferred the accuracy of looking food up in the database; and some participants combined the two approaches. Participant entries consisted primarily of the food group components. The number nutrient component entries was smaller than expected. Finally, participants expressed frustration when they could not retroactively create food entries. The follow-up interviews indicate that participants want a tool that is flexible in regards to the user's schedule. Sometimes their schedule allows them to focus on the food diary right after eating, while other times the user needs to go back in time. 

Future work of the POND tool falls into two categories: improving the design, and further evaluating how different populations respond to the POND approach to self-monitoring. The design of POND could be improved by incorporating more contextual awareness, particularly in terms of location and schedules. It should also be easier for users to create food entries for food eaten earlier. Finally, the design may benefit by focusing on the distinction between nutrient-dense versus energy--dense foods, specifically in regards to specifying serving size. 

Future studies with POND should focus on two areas. The first is around the characterization of users. The trans-theoretical model can be used to identify how motivated users are, and there should be a more detailed identification of user goals. A second area of evaluation should focus more on specific patterns of detailed versus overview self-monitoring. For example, participants may spend one week adhering to the detailed self-monitoring approach, then two or three weeks with an overview self-monitoring approach (the +1 buttons).  The detailed self-monitoring may establish a strong understanding of the user's baseline behavior, provide education around nutrition values for foods the user currently eats, and provide a baseline for setting goals for change. The overview self-monitoring requires less time to execute, but can serve as a reminder of current goals and progress. These two evaluations are first steps toward a larger, long-term, randomized control trial to assess the impact on larger populations. 

This dissertation has identified ways to potentially characterize populations for food diary evaluations. Through the three projects, participants could be characterized in terms of goals, motivation, and resources. Goals includes what the individual wants to achieve. Many of our self-selected participants indicated that while they would like to lose weight, they would be happy to simply improve some of their nutrition behaviors. Others indicated that these goals changed over time: sometimes they were interested in losing weight or preparing for athletic events, while others they just wanted to stay healthy.   Motivation refers to how motivated the individual is. Some participants acknowledged that they would like to lose weight, but it was not currently a high priority. Finally, resources includes how much time and money a person is willing to put into those attaining their goals. This is different than the goal or motivation. All three work together to effect an individual's self efficacy. 

We see some indication of these differences of goals, motivation and resources in the participant feedback in the BALANCE studies. The in-lab study indicated that participants with different kinds of goals may prefer different self-monitoring approaches. Metrics from the use of the POND software both in the lab and \textit{in situ} indicate that these preferences appear consistent with changes in goals, motivation and resources. This is also consistent with feedback from participants who indicate that their preferences and willingness to use different tools for self-monitoring dietary intake has changed overtime. Based on this, there could be value in future work that focuses on a more rigorous characterization of motivation, goals and resources in regards to nutrition behaviors. 


Evaluation is a recurring theme in the projects in this dissertation. In BALANCE, the evaluation focused primarily on the design and performance of the system. This was followed up by a study that compared the usability and presumed utility of multiple food diary designs. Finally, the POND work combined usability tests and an \textit{in situ} evaluation provide a more comprehensive analysis of the software. In all of these projects, I collected a wide range of metrics and provided a detailed analysis. The description, collection and analysis of these metrics can be used in future work: User research requires the collection of many different metrics because it is usually unclear \textit{a priori} which metric will be most informative. Publishing these results allows researchers to compare new systems with previous systems, with the ability to identify and quantify differences. 

The creation of effective self-monitoring of dietary intake tools requires strong collaboration between domain experts (e.g., nutritionists, epidemiologists and behavior change experts), technology experts and user experience researchers. User experience (UX) researchers mediate the design and development process of new tools. They work with domain experts to understand and define theory-informed goals (changing health behaviors) and technology experts to understand capabilities or restrictions of technology. UX researchers then work with target users to understand what barriers they may face, and how the new technologies can overcome these barriers. The UX researcher is important in this role because of their unique expertise in understanding users. 


It is important for Researchers sometimes need highly accurate, very detailed information about what people eat. These areas of research were covered in Chapter \ref{cha:relatedWork}. The self-monitoring tools for this research should support the goal of the research, but with the understanding that the it might not match the researcher goals. We should not  assume that what is appropriate for researchers who need accuracy is what is appropriate for users. 

Something I hope other researchers can take from this work is an awareness of assumptions that are made about the design and use of food diaries. This area of research has been avoided in the technology community because of its complexity. As we saw with the BALANCE study, building a traditional food diary is not a trivial project. 


The research area of self-monitoring of dietary intake is complex, interdisciplinary, and will have an impact on the health and wellness of millions of people. When technology, nutrition, and user experience experts come together, we will be able to design and build effective and enjoyable tools to support healthy behaviors. 

