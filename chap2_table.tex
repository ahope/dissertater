
% Table generated by Excel2LaTeX from sheet 'chap2'

% \begin{sidewaystable}[htbp]
\begin{landscape}
\tiny
  \centering
    
\begin{longtable}{l
p{1in}
p{0.5in}
p{0.35in}
p{1in}
p{0.35in}
p{1in}
p{1in}
p{1in}}

       \toprule
    \bf{Reference} & \bf{Platform \& Instrument } & \bf{Database Information } & \bf{Number of Participants } & \bf{Target Population } & \bf{Duration } & \bf{Results } & \bf{Challenges } & \bf{Goal or purpose} \\
    \midrule \endhead
\cite{glanz_improving_2006} & 
PDA. \newline
Custom software. \newline 
Tracked fat, servings of fruit, veggie \& grains. Provided customized (paper) report after every week. & 
300 items (custom built from program materials) & 
33    & 
Women in Diet Modification arm of WHI\footnote{Women's Health Initiative: Highly motivated to adhere to target diet plan that is focused on reducing dietary fat intake.}. & 
1 month & 
Entries made a mean of 5 days/wk. \newline 
Half of ppts made entries 6-7 days/wk. \newline 
On 62\% of days, entries were made 3+ different times. &       
None reported.& 
Support participants in following a particular diet plan \\
\cite{burke_effect_2011} &
PDA (DietMatePro) with customized feedback component (FB)\newline
PDA w/o FB\newline
Paper report (PR)\newline &
6000 items (USDA) &
210   &
Overweight or obese (BMI)\newline No medical conditions &
6 months & 
PDA+FB=90\%, PDA=80\%, PR=55\% \newline
Adherence definition: Ppt was adherent for the week if they made some recordings each day and recorded > 50\% of target calorie goal for the week &
None reported. &
As part of a larger behavioral support program that included group sessions   \\
\cite{acharya_using_2011} & Same as above & Same as above  & Same as above   & Same as above & Same as above &
PDA groups: increased fruit and vegetable intake, decreased refined grain and fat intake. \newline
Across all groups, frequency of self-monitoring correlated with sugar intake (more frequent records, higher sugar intake)  &
None reported.  &  
Evaluate impact of self-monitoring on the quality of diet. \\
\cite{jarvinen_hyperfit:_2008}& Camera phones. \newline
Barcode scanning supported with additional macro lens. \newline
Custom developed web \& mobile apps.  & 
2500 items & 
97 individual users;\newline
5 groups (39 participants total);\newline
9 nutritionists;\newline & 
Trial 1 \& 2: people interested in weight management; \newline
Trial 3: tool for support in a weight management group;\newline 
Trial 4: as a tool for nutrition counseling. & 
Trial 1 \& 2: 2 weeks; \newline
Trial 3: Groups: tasks of tracking for 2-3 days at a time.\newline
Trial 4: 3 wks;  & 
Individuals: overall positive, with food \& exercise diaries being most useful. \newline
Nutritionists: overall positive \& useful, food diary most used. \newline
Groups: positive impression for both members and instructors. &  
Tool is 	``effort-demanding''; provides ``huge amount of information'' for those who are already motivated. Ability to use more usable and extensive nutritional data. Smartphone \& barcode scanning technology in infancy. Creating records in food \& exercise diary were too time consuming and challenging. & 
Feedback about the use of the entire system (nutrition information, self-monitoring capabilities, feedback from virtual trainer, more)\\
\cite{mattila_mobile_2008} & 
Symbian S60 smartphones. \newline
Custom software (Wellness Diary): Food tracking consisted of classifying as light snack, heavy snack, light meal, heavy meal; optional entry of calories.   & 
none  &
29    & 
Overweight, used S60 phone &
3 months & 
LOSERS made more entries in all categories than OTHERS group; 79\% thought it would help them lose weight. 64\% wanted to continue using it &
Importance and input frequency of food decreased over the course of the study. People felt it important to be careful in observing at the beginning, but less so later.;Importance of recording weight increased over the study. &
Test ability of WD to support CBT-based weight management. Participants educated about using CBT for weight loss. \\
\cite{kozakai_dietary_2006} & 
Camera phone. \newline
Custom software; included scale, blood pressure meter, diet diary & 
607 w/o barcode, 805 w/ barcode &
 1     & 
Grad student volunteer & 
2 months &
       & 
Food sometimes was not in the database.  & Proof of concept that these tools can be used to track health indicators over time. \\
\cite{reddy_image_2007} &
Nokia N80 phone\newline 
Custom software (DietSense): captured images, audio, context \& location every 10 seconds &
None     &
6     &       
Users & 
2 weeks & 
How to improve automatic capture of meal photographs. \newline 
Identified potential areas of concern around automatic capture.   &
Need better tools for distinguishing between meal and non-meal photos.  & 
Evaluate potential usefulness for DietSense, collect data to inform future development. \\
\cite{silva_sapofitness:_2011} & 
Android. \newline
Custom softwared.  & 
Not reported &
``Several users'' &       & 
``Several weeks'' & 
``There was pretty good feedback from the users''& 
None reported.  &
Prove ability of SapoFitness to motivate weight loss. \\
\cite{tsai_usability_2007} & 
Cell phone (Motorola RAZR)\newline
Custom software (PmEB); &
750 items & 
15    & Clinically Overweight (BMI$\ge$ 25) &
1 month &
Food entry was challenging, due to limited database, inadequate food search algorithms, inability to enter fractional servings. \newline
Users disliked the prompts      &
1-prompt group used app 6 times per day; Paper group used record 5.5 times per day; 3-prompt group used app 3 times per day.\newline
1-prompt group made entries 99.1\% of days; 3-prompt group 96.4\% of days; Paper group 60.7\% of days.        &
Compare use of PmEB with 1 daily reminder, PmEB with 3 daily reminders, and use of paper records.   \\
\cite{long_effectiveness_2012} &
Camera phones (no custom software), MyPyramid tracker (website). & 
6000 items (USDA) & 69    &
 Students &
 3 days &       & 
Photos as memory prompts were helpful but not perfect. Serving sizes were still an issue. & 
Evaluate effectiveness of cell phone photos as an effective memory prompt when using MyPyramid tracker \\
\cite{atienza_using_2008} & 
PDA. \newline 
Custom software. 43-question assessment (triggered 2x/day, provide individualized feedback, goal-setting and support) &
None  &
 27    &
 Adults $\ge$ 50yrs &
 8 weeks &   
In the PDA group, intake of vegetable and dietary fiber from grains increased more.\newline
Ppts completed 51\% of the assessments.; Week 1=75\%, Week 8=40\%. &  
& 
Compare impact of PDA versus education intervention on increasing vegetable \& whole grain intake.\\
\cite{yon_personal_2007} & 
PDA. \newline
Calorie King's Diet Diary software. &
Not reported in paper; current CalorieKing Diet Diary states 50,000 records &
 (71 PDA, 115 paper control) &
 Overweight, participating in weight control program &
 6 months &
No significant difference between PDA and paper use. Overall, individuals who self-monitored more lost more weight. &
 26\% of PDA users liked the ``ease of use of entering food and exercise data'', 44\% reported they disliked the PDA software (because they could not find food they eat or  could not see the screen). Ppts had trouble navigating the software and finding foods. &
 Evaluate use of PDA in a behavioral weight loss program (including weekly meetings; positive, personalized feedback; and technical support). \\
\cite{fukuo_development_2009} &
PDA.\newline
Custom software.       &
423 food \& drink photos  &
 44 w/o diabetes, 16 w/ diabetes&
 Convenience sample &
 7 days &
 For both groups, PDA entries agree with 24-hr recall &
 None reported.      &  Evaluate effectiveness of food photographs for aiding in serving size specification in a PDA-based food diary. \\
    \bottomrule

%   \end{tabular}%
\end{longtable}
  \label{tab:addlabel}%
%\end{sidewaystable}%
\end{landscape}
