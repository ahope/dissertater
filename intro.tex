\chapter {Introduction}

This dissertation explores the how people use mobile phone food diaries to self-monitor their dietary intake. In this chapter, I identify some factors that influence what people eat; provide background on how what people eat impacts their health; and introduce how technologies that influence dietary intake can impact society. 

\section{Introduction}
Globally, the incidence of lifestyle diseases such as obesity, diabetes and cardiovascular disease are increasing. One-third (34\%) of adults were overweight or obese in 2008 \citep{kimokoti_diet_2011}. The prevalence of diabetes for all age groups, worldwide, is expected to rise from 6.4\% in 2010 to 7.7\% (439 million people) in 2030 \citep{Shaw20104}. Researchers believe this to be an underestimate. Cardiovascular disease (CVD) is the number one cause of death globally. 17.3 million people died from CVDs in 2008, accounting for about 30\% of worldwide deaths \citep{WHO_CVD_2011}. Additionally, it is estimated that at least 50\% of cancers are preventable by encouraging healthy behaviors and discouraging unhealthy practices. Cancer accounted for 7.6 million deaths in 2008. The estimated number of new cancer cases in 2030 is 21.4 million, with 13.2 million deaths \citep{american_cancer_society_global_2011}. 

Overweight and obesity are major risk factors for cardiovascular disease (CVD), Type 2 diabetes mellitus, and certain forms of cancer \citep{guh_incidence_2009} \citep{calle_overweight_2004}. 

Lifestyle diseases are so named because they are associated with how a person lives and the choices they make. Lifestyle choices include what and how much to eat, physical activity levels, alcohol consumption, tobacco use, sleep habits, and stress management. Obesity, diabetes, cardiovascular disease and some cancers are all considered lifestyle disease. Generally, researchers believe that lifestyle choices not only impact the incidence or likelihood of developing these diseases, but can also help to treat or reduce the impact of these diseases. 

Researchers believe that a primary cause of obesity is an imbalance of energy intake and expenditure: People eat too many calories while not burning enough calories. While obesity is a risk factor for CVD, diabetes and cancer, it is believed that these diseases are also impacted by what people eat. In addition to balancing caloric intake, the American Heart Association recommendations for preventing cardiovascular disease includes consuming a diet rich in fruits and vegetables, choosing whole grain--high fiber foods, limiting saturated fat intake, and reducing consumption of added sugars and alcohol \citep{Lichtenstein2006}. Similar recommendations are made for preventing diabetes \citep{Hu2001, Thomas2012}. 

Traditionally, researchers believe what people eat is due to biological, psychological, behavioral and social factors. \citet{furst_food_1996} found that food choice included ``ideals, personal factors, resources, social contexts and the food context''. Food choice incorporates ``value negotiations and behavioral strategies''. Values reflect taste and texture of foods, cost, health and nutritional concerns, availability, and social context. 

More recently, researchers believe that the built environment also impacts what people choose to eat. Built environment includes urban design, land use, availability of public transportation, and activity options in an area \citep{Booth2005}. This also reflects the number of fast food restaurants, convenience stores, bars, food distribution programs with high-fat foods, and concentrated media marketing. Poorer neighborhoods have fewer supermarkets and more places to consume alcohol than wealthier neighborhoods \citep{Morland2002}. This impacts the availability of healthy food. 

The food industry plays a role in the over consumption of food in America. Serving sizes have increased over time \citep{Nielsen2003}: Commercial serving sizes are at least twice the size and as much as eight times the size of standard serving sizes defined by the USDA and FDA \citep{Young2003, Nielsen2003}. Servings provided by fast--food chains are now two to five times original sizes \citep{Young2003}. Calories provided by the US food supply increased from 3200 in the 1970s to 3900 in 1990s \citep{Nestle2002}. When more food is available, manufacturers need to compete harder for their share of consumer dollars. The increased marketing encourages people to eat more. 

However, the external environment takes a long time and many resources to change. Providing support for individuals to insulate themselves from the environment and change their own behavior could have a more immediate impact.  \citep{Hill2006}
Research indicates that self-monitoring of dietary intake does help change behavior \citep{burke_experiences_2009,burke_self-monitoring_2005, burke_self-monitoring_2011}. 

Many people are not well aware of what, why or how much they eat.   While people tend to underestimate energy intake \citep{Schoeller199518,Black1991583,Livingstone1990}, in general they are unsure of how many calories they burn in physical activity (energy expenditure) \citep{Prince2008}. This tendency results in an unawareness of the actual overall balance of caloric intake and expenditure, which results in weight gain. 

People underestimate energy intake for a variety of reasons, including a lack of awareness of how many calories are in food \citep{Zegman1984}; what an appropriate serving size is \citep{Young2009, beasley_accuracy_2005}; erroneous estimates of how much of a food has been eaten \citep{beasley_accuracy_2005}; and neglecting to consider calorie dense condiments or additions to a food \citep{Zegman1984}. This effect is magnified when a person does not document what they eat as soon as they eat it, and time passes between consumption and recording \citep{Baranowski1994}. Additionally, although reflecting on food intake in the past (e.g., earlier in the day) may help raise an individual's awareness of calories in given foods, it does not enable them to make good decisions in the moment. Timely access to calorie data could help people not only capture and understand how many calories they have already consumed for the day, but also make better decisions going forward. 

People also underestimate the number of calories they burn over time. An estimate of energy expenditure by a person in a given time period is the sum of three items: the base metabolism for that individual (how many calories their body consumes for basic bodily function), the amount and intensity of intentional exercise (e.g., going for a run) and the amount and intensity of unintentional exercise (e.g., how much walking versus sitting a person does). It is relatively easy for people to measure their intentional exercise, as it's usually well-defined. However, humans tend to vastly overestimate their unintentional exercise \citep{Duncan2001}. Readily available technology such as wearable sensors can keep track of general activity for an individual, allowing them to more closely track energy expenditure over time. 

The field of persuasive technology aims to use computers and technology to persuade or nudge people to change their behavior, in a way that they want to change \citep{Fogg2002}. 

Ubiquitous mobile devices such as smartphones have been considered ideal platforms for administering behavior change programs, especially around eating behaviors. They enable kairos-- the availability of relevant information and decision support at the time of need. That is, these devices allow people to self-monitor what they eat by providing a means to capture what has already been eaten, and looking up caloric values for prospective foods. This allows people to make an informed decision. However, formal research and informal feedback indicates that people have difficulties adhering to the use of mobile-phone food diaries for extended periods of time. 

This dissertation focuses on the use of mobile technologies to support people in becoming more aware of their nutrition behaviors and make better choices. 


\section{Roadmap}
Chapter \ref{cha:relatedWork} describes background in psychology and epidemiology that focuses on behavior change and how to support it. It also reviews related work and how it differs from the work presented in this dissertation.  Chapter \ref{cha:cont1} describes the BALANCE project, which explored one approach to supporting people in self-monitoring food intake and physical activity. Chapter \ref{cha:cont2} describes background and a study that shows that there could be benefit in designing a food diary that requires less input and thinking. Chapter \ref{cha:cont3} describes the design of POND, which is a food diary based on the results of the study in Chapter \ref{cha:cont2}. Chapter \ref{cha:cont4} evaluates the POND food diary. Chapter \ref{cha:futureWork} concludes this dissertation with a summary and discussion of future work. 
