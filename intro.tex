\chapter {Introduction}

This dissertation explores the how people use mobile phone food diaries to self-monitor their dietary intake. In this chapter, I provide some background on how what people eat impacts their health. I also identify some factors that influence what people eat. This chapter concludes with an overview of the rest of the dissertation. 


\section{Introduction}
Globally, 34\% of adults were overweight or obese in 2008 \citep{kimokoti_diet_2011}. While this has been a concern particularly in Westernized countries, the epidemic is spreading globally: Nauru in the Oceania reported the greatest gain in BMI globally \citep{finucane_national_2011}. Overweight and obesity are major risk factors for cardiovascular disease (CVD), Type 2 diabetes mellitus, and certain forms of cancer \citep{guh_incidence_2009} \citep{calle_overweight_2004}. 

Obesity, CVD, diabetes, and some forms of cancer can be prevented and treated by what people eat. 

There are many reasons why people eat what they do. Some of these reasons are internal: there are social situations, convenience, cost, tradition. 

External influences also impact what people choose to eat. Research points to ``obesogenic'' environments. Serving sizes have grown over the years. The use of high fructose corn syrup, refined grains, and other highly processed foods in our food supply has grown. More food is available in the world, primarily from manufacturers who need to sell it. Food manufacturers want to make money, so they advertise a great deal. I haven't even touched on the physical activity side of the environment. 

The external environment takes a long time and many resources to change. Changing ourselves is more immediate. 

The personal responsibility approach has flaws. Given all that is in the environment, focusing too much on the individual can be demotivating. 

The "personal responsibilty" view of dietary intake has pros and cons. But, nonetheless, many people believe that there could be some value. Plenty of research indicates that self-monitoring of dietary intake does help change behavior. And there is lots of money in it-- many many apps and websites are dedicated to helping people monitor their health and wellness behaviors. 

The smartphone was considered a the killer platform for administering behavior change programs, especially around eating behaviors. They enabled kairos-- the availability of relevant information and decision support at the time of need. However, formal resaerch and informal feedback indicates that people are not able to keep using it. These tools are hlepful for a short time, then the user gets bored or disillusioned. 

This led to the BALANCE project, where we combined physical activity detection with dietary intake monitoring to  encourage users to self-monitor. 

The research that indicates self-monitoring helps people to lose weight is not detailed enough to understand what aspects of self-monitoring are necessary. Additionally, this research is usually done on highly motivated participants. 


Self-monitoring, defined as the process of observing and recording target behavior, has been identified as a key component of behavior change in general \citep{kanfer_self-monitoring:_1970} and for weight loss in particular \citep{michie_effective_2009} \citep{burke_effect_2011}. Of the people who have lost weight and kept it off (as registered with the Weight Loss Registry), the key behavior that correlates highly with losing and keeping weight off is the practice of self-monitoring both food (energy) intake and physical activity (energy expenditure) [ref]. In general, people tend to underestimate their energy intake, while overestimating how many calories they burn in physical activity (energy expenditure) [ref]. This tendency results in an unawareness of the actual overall balance of caloric intake and expenditure, which results in weight gain. 

People underestimate energy intake for a variety of reasons, including a lack of awareness of how many calories are in food [ref]; what an appropriate serving size is [ref]; erroneous estimates of how much of a food has been eaten [ref]; and neglecting to consider calorie dense condiments or additions to a food [ref]. This effect is magnified when a person does not document what they eat as soon as they eat it, and time passes between consumption and recording [ref]. Additionally, although reflecting on food intake in the past (e.g., earlier in the day) may help raise an individual's awareness of calories in given foods, it does not enable them to make good decisions in the moment. Timely access to calorie data could help people not only capture and understand how many calories they have already consumed for the day, but also make better decisions going forward. 

People also underestimate the number of calories they burn over time. An estimate of energy expenditure by a person in a given time period is the sum of three items: the base metabolism for that individual (how many calories their body consumes for basic bodily function), the amount and intensity of intentional exercise (e.g., going for a run) and the amount and intensity of unintentional exercise (e.g., how much walking versus sitting a person does). It is relatively easy for people to measure their intentional exercise, as it's usually well-defined. However, humans tend to vastly underestimate their unintentional exercise [ref]. Readily available technology such as wearable sensors can keep track of general activity for an individual, allowing them to more closely track energy expenditure over time. 

\section{Roadmap}
Chapter \ref{cha:relatedWork} describes related work {}. Chapter \ref{cha:cont1} describes the BALANCE project, which explored one approach to supporting people in self-monitoring food intake and physical activity, key indicators in health and wellness. Chapter \ref{cha:cont2} describes background and a study that shows that there could be benefit in designing a food diary that requires less input and thinking. Chapter \ref{cha:cont3} describes the design of POND, which is a food diary based on the HEI. Chapter \ref{cha:cont4} evaluates the POND food diary. Chapter [ref] concludes this dissertation with a discussion of future work. 
