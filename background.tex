\chapter{Background}
\label{cha:background}
This dissertation is focused on the use of technology to support self-monitoring to effect behavior change. This chapter summarizes background information related to processes of change. This includes models of human behavior, learning and change. Within these models, I examine the role of self-monitoring. This background will highlight the role and importance of self-monitoring as a tool for nutrition behavior change. Included in this background is persuasive technology, which uses the aforementioned theories to define strategies and tactics of employing technology to nudge behavior. 

\section{Introduction}
Self-monitoring has long been acknowledged as a means to change one's behavior. Additionally, with the increase in accessibility of handheld computing devices and network connectivity, self-monitoring tools and software has proliferated. I will present some examples of the many forms of self-monitoring that are available today in the next chapter, but in this chapter I focus on the theory behind self-monitoring. 

Self-monitoring is both a source of data about a target behavior and a therapeutic process, in that it usually causes a change in the observed behavior. Smoking [ref], exercise [ref], and depression are all 

Over the years, self-monitoring tools have been used to modify behavior, but in regards to the use of self-monitoring tools for changing nutrition behaviors, we do not know how or why they work or don't work, or what features are helpful or not helpful. 

For this dissertation, it is important to understand the theoretical basis of the role that self-monitoring tools play in the greater behavior change process. 

This chapter is organized into four sections. First, I introduce some key theories of mind and behavior. For this dissertation, I include one that focuses on individual, internal behavior (Theory of Planned Behavior) and one that focuses on the relationship between the individual and their larger environment and community (Social Cognitive Theory). Second, I describe two therapy approaches commonly used by experts to support individuals changing their nutrition behaviors, Motivational Interviewing and Cognitive Behavior Therapy. These therapies are informed by theories of mind and behavior. Third, I describe some cross-cutting constructs that pertain to many behavioral theories and contribute to the planning and executing of the therapies. Finally, I describe some tools and technologies that can support the therapeutic process. 

\section{Theories of Behavior}
Humans have been philosophizing on why we do what we do since the beginning of time. It is outside the scope of this dissertation to enumerate all theories of behavior, many of which build on each other. However, it is important to understand the theoretical basis behind the design of tools presented in this dissertation. 

Behavior theories range from focusing on internal forces to larger external forces, including societal and historical factors. In this dissertation, I focus on two theories. These two theories are the basis for much of the related work in this area. The first is the Theory of Planned Behavior, which focuses on an individual's internal conception of behavior. The second is Social-Cognitive Theory, which accounts for how an individual's environment impacts one's behavior and ability to change. 

\subsection{Intent to change: Theory of Planned Behavior} 
The Theory of Planned Behavior (TPB) is a model of individual behavior that assumes a person is a rational actor. It builds off the Theory of Reasoned Action (TRA). TRA states that a person's behavior is determined by behavioral intention. Behavioral intention depends on attitude (individual beliefs about a behavior) and a subjective norm. The subjective norm stems from the individual's normative beliefs. Normative beliefs are defined as whether influential parties approve or disapprove of the behavior in question. 

Using TRA to predict or explain behavior consists of calculating both \textit{attitude} and \textit{normative beliefs} of an individual. A measure of \textit{attitude} is calculated by asking the individual a series of questions that measure the likelihood that a target behavior will result in a specific outcome and evaluate how desirable the given outcome is. The potential outcome of a single behavior is defined by combining the likelihood and desirability. An overall behavior change is usually composed of many small behaviors. Therefore, the attitude toward the overall behavior change is composed of the attitude toward each behavior that composes the change. 

Similarly, \textit{normative beliefs} are calculated by first identifying relevant referents for the individual. A referent is a person of influence in the individual's life. This could be anyone, including a member of the immediate family, a friend or colleague, someone in a position of power such as a boss or clergy member, or a health care provider. For each of the identified referents, the individual reports feelings about what that referent feels about the individual performing the target behavior. A statement reflecting this notion is ``My doctor feels strongly that I reduce my saturated fat intake''. Then, for each referent, the individual reports the amount of influence that referent has, e.g. ``I strongly agree that I want to do what my doctor suggests''. These two reports are combined for that referent. Then, the results for each referent are combined to represent overall normative beliefs. 

The Theory of Planned Behavior (TPB) extends TRA by including a component of volitional control. Volitional control captures the individual's ability to exercise control over the target behavior. Volitional control is included in the model by perceived behavioral control. If the individual perceives a great amount of behavioral control, then motivation (as characterized by TRA) drives the individual's performance of the target behavior. If the individual perceives little control over their ability to execute the target behavior, the individual expends less effort in pursuit of the target behavior. Perceived control depends on control beliefs and perceived power of each belief. 

\subsection{Context-aware change: Social Cognitive Theory}
Social Cognitive Theory \citep{bandura_health_2004, bandura_self-efficacy:_1977} is based on reciprocal determinism: a person�s behavior, environment and inner qualities continuously interact. This theory includes constructs that address how people learn, how environments impact behavior, and how people regulate their own behavior. 

One of the key contributions of SCT is \textit{observational learning}: the belief that people can learn by observing others learning and experiencing \textit{vicarious reinforcement}. This is in contrast to the more traditional belief of \textit{operant learning}, which dictates that an individual needs to execute a behavior and be directly rewarded for it in order to learn. A factor in using observational learning to change one's behavior is the individual's perception of how similar the observed situation is to their own situation. That is, if Person A observes that Person B engaged in specific behaviors to stop smoking, Person A will believe that applying similar behaviors will have a similar outcome if Person A observes that Person B has similarities in key personal and environmental factors. Observational learning is particularly relevant when considering health behaviors because the reward or reinforcement for executing healthy behaviors is frequently not immediate, but emerges over time. 

The role of environment and perception of environment are also important SCT. Environment includes all factors external to the individual, including interpersonal relationships. When considering nutrition behaviors, environment may dictate what foods are available to an individual, for example due to seasonal availability or what is available to purchase at a grocery store or cafeteria. The social environment may dictate what foods are appealing to choose or the ability to request specific foods (e.g., whether a child can ask an adult for special foods). 

Self-efficacy is the belief in one�s abilities to overcome specific challenges associated with a target behavior. Self-efficacy derives from the earlier mentioned interaction among the individual's behavior, the environment, and the individual's inner qualities. When the individual executes an action within the environment, the result either meets expectations or not. If the individual's self-efficacy was high around this particular action, an unanticipated result may not have much of an impact. However, if the individual's self-efficacy was low, an unanticipated result could substantially negatively impact the individual's belief around that behavior, environment and ability to navigate it. 

Self-efficacy is traditionally measured by self-report. To develop self-efficacy measurements for a particular domain, researchers use open-ended approaches to identify common challenges and barriers to the problem. They then develop a series of statements of the form ``How confident are you that you can [achieve goal] even though [challenge]?'' with a 4-unit response scale ranging from ``Cannot do it'' to ``Highly certain can do''. An example of a statement is ``How confident are you that you can stick to a healthy eating plan after a long, tiring day at work?''. Research shows self-efficacy measures based on self-report indicate adherence to strategies to change behavior \citep{nothwehr_self-efficacy_2008}. While short-term studies cannot prove behavior change, self-efficacy measures provides valuable feedback about whether an intervention is supporting adherence to behavior change strategies, and indicate whether participants complete the study with an intention to continue.


\section{Supporting Change: Therapy Approaches}
One of the reasons humans have been consumed by thinking about theories of mind, thought, and behavior is because the human condition includes being unsatisfied with the status quo. More recently in human history, the role of therapist has established itself in our society. Therapists use an understanding of mind and behavior theories to help individuals solve problems. 

In this section, I introduce two therapies that are commonly used to address behavioral change around nutrition and health behaviors. These therapeutic approaches represent the gold standard of care for helping people to change their behaviors. Understanding these therapies helps to either evaluate the role of technology in supporting the therapeutic process, or using the therapeutic process to inform the design of technology to support change. 

\subsection{Motivational Interviewing}
Motivational interviewing (MI) \citep{miller_motivational_2002, Miller2002} is a process practitioners use to identify what areas of change an individual is ready for. It is a technique employing goal setting, feedback mechanisms, and self-monitoring. There are three key elements: collaboration rather than confrontation; evoking ideas rather than imposing ideas; and autonomy rather than authority. In addition are four principles to follow: express empathy, support self-efficacy, roll with resistance, and ``develop discrepancy'' (or identify cognitive dissonance). 

MI has a strong record of impact for many types of behavior change \citep{Rubak2005}. It is particularly effective and well respected in the area of nutrition behavior changes \citep{abraham_taxonomy_2008, michie_effective_2009, greaves_systematic_2011}. 


\subsection{Reframing change: Cognitive-Behavioral Therapy}

Cognitive-Behavioral Therapy (CBT) is based on a cognitive model called \textit{emotional response}, in which thinking influences behavior. Under this premises, changing one's thinking changes one's behavior. Some therapies that fall under the CBT umbrella focus on emotional states or feelings, such as depression or anxiety. Other therapies use a similar approach to change overt behaviors such as nutrition or smoking. 

CBT therapy focuses on educating the individual about different cognitive patterns and how those cognitive patterns manifest in that individual. The therapy process is based on the Socratic Method, where the therapist asks questions about the individual's thinking. This teaches the individual to eventually ask similar questions of themselves. 


\section{Cross-cutting Constructs}
So far, I have discussed behavioral theories and therapies to support behavior change. These have each been distinct in that there are clear boundaries between where one theory or therapy. Each theory and therapy was self-contained. They are not contradictory, but they do not specifically build on each other. In this section, I introduce some constructs that apply to many different theories and therapies. 

\subsection{The process of change: Transtheoretical Model}
The transtheoretical model (TTM) \citep{prochaska_transtheoretical_1997} reflects that people are in varying stages of change in relation to a given behavior. It arises from study of a theoretical intervention therapies, such as psychoanalysis, gestalt, and cognitive therapies. Across all these theories, Prochaska et al determined that intervention therapies had no time construct. He interviewed people who were not familiar with intervention therapies but had successfully changed a key behavior. He found that change was a process that unfolded over time. This process could be broken into stages, each with certain characteristics. Further, constructs from some theories were applicable in some stages of change, while a seemingly contradictory theoretical construct was appropriate for a different stage of change. A key finding from the TTM inquiry is that no single theory can fully describe and account for the complexities of behavior change. 

The stages include precontemplation, contemplation, preparation, action, and maintenance. A person in the stage of precontemplation does not intend to take action around the target behavior within the next six months. That person moves to contemplation when they intend to take action within the next six mohts, and then into preparation when they intend to take action within thirty days. People in the preparation stage usually have taken some steps toward the change. Action reflects that the person has engaged in the new behavior for less than six month. Finally, maintenance is having continued the new behavior for more than six months. 

Two constructs impact progression from one stage to the next: decisional balance and self-efficacy. Decisional balance is a comparison of the benefits of changing to the costs of changing. To move from the preparation to the action stage, the benefits need to increase by one standard deviation, or the costs need to decrease by half a standard deviation. That is, the person needs to either identify many more benefits to changing, or reduce the impact of a few barriers. Self-efficacy is a measure of how confident an individual is about their ability to continue the target behavior in challenging situations. 

Understanding the TTM is important for designing behavior change interventions because different theoretical constructs and processes apply to different stages of change. For individuals in precontemplation, consciousness raising and environmental reevaluation are key processes. These processes focus on strategies that target education of the individual, or changing how they view themselves in relation to the target behavior. In the action stage, processes such as counterconditioning and helping relationships support continuation of the target behavior. Counterconditioning focuses on how to replace unhealthy behaviors with target behaviors. Helping relationships includes building relationships that support the individual in the change. 


\subsection{Motivating change: Goal setting and tending}
Here I talk about Locke and Latham \citep{locke_building_2002} and Consolvo and Landay \citep{consolvo_goal-setting_2009}. Goal theory. How to define goals. How to support progress toward goals. 

Goal setting theory \citep{locke_building_2002} focuses on the characteristics of effective goals, including the process through which they are defined, who is involved in the definition, and how challenging the goals are. Goals can be used to motivate change by directing attention and action toward specific target for performance. When people are given high goals and are able to negotiate the time frame, they work harder to meet those goals. In contrast to exhorting people to ``do their best'', setting and tending goals motivates people to perform better overall. 

There are three mediators to goal performance: goal commitment, feedback, and task complexity. Goal commitment includes both importance and self-efficacy. Importance may be either internally or externally driven. Goals set by an external mediator (e.g. boss, coach) benefit from including the individual in the goal-setting process due to the cognitive engagement: providing an explanation for why the goals are important. Self-efficacy reflects how capable the individual feels about attaining the goal. Increasing self-efficacy can increase commitment to attaining the goal. Feedback is important to allow individuals to track progress toward goal attainment. Feedback allows individuals to modify their approach as necessary. Finally, task complexity is key to the attainment of goals. For challenging tasks, individuals perform better on tasks that focus on a specific learning goal as opposed to a general performance goal. Additionally, breaking a complex, challenging task into appropriate sub-tasks helps people to perform those tasks and successfully achieve the overall complex task.  

When people with high self-efficacy set goals, they set higher or more challenging goals than those with low self-efficacy. They also ``are more committed to assigned goals, find and use better task strategies to attain the goals, and respond more positively to negative feedback than do people with low self-efficacy'' \citep{locke_building_2002}. 



Goal setting theory is consistent with SCT, in that they both incorporate the importance of goals and self-efficacy.

\section{Supporting Change: Tools}
I have now introduced theories of behavior, therapies designed to support behavior change, and constructs that describe or mediate the entire behavior change process. In this section, I describe tools that can support behavior change. Per the focus of this dissertation, I address self-monitoring and the use of technology to support change. 

\subsection{Self-Monitoring}
Self-monitoring of dietary intake is the process that an individual uses to keep track of what they eat. The self-monitoring process is impacted by both the tool or instrument used for self-monitoring and the individual doing the self-monitoring.  The design of a self-monitoring tool impacts the self-monitoring process in a number of ways. First, the tool serves in a focusing role: it specifies what the individual should be monitoring. In the realm of nutrition, it could be calories, a rating of the healthiness of a meal, or many other aspects of the food. Second, the tool can help by clarifying goals and tracking progress toward those goals. Finally, the design and implementation of a self-monitoring tool can make it easier or more challenging to adhere to the self-monitoring process. 

In regards to the individual, the process of self-monitoring involves cognitive processes. The cognitive processes children use when recalling food have been identified as attention, perception, organization, retention, retrieval and response of information (\citep{Baranowski1994}). The individual needs to attend to the food that they are eating, at the time they eat it, in order to perceive it. The food information then needs to be organized and stored for retention. Then, the food information needs to be retrieved and formulated into a response or action (e.g. creating a record).  

In addition to the low-level cognitive processes that impact self-monitoring, the ability or willingness to continue self-monitoring involves individual goals, motivation, and resources. Goals and motivation were discussed earlier in this chapter. Resources are accounted for in TPB, as the construct of volitional control. An individual's resources, or lack of, can impact the self-monitoring process by impacting their ability to monitor consistently. Lack of time is frequently cited as a barrier to self-monitoring of dietary intake. In addition, one could imagine a scenario with a mobile-phone based self-monitoring tool, when the phone runs out of battery, or perhaps does not have a network connection. 

Research consistently supports that self-monitoring helps to mediate behavior change. Indeed, \citet{kanfer_self-monitoring:_1970} states ``self-observation is an initial step to self-directed behavior change''. Much research in this area falls outside the scope of this chapter, but in the domain of weight-loss, people who adhere to consistent self-monitoring of dietary intake lose more weight and have better indicators of related behaviors. Burke et al and the PREFER trial. WHI. \citep{greaves_systematic_2011}. 

It is unclear how the process of self-monitoring impacts behavior change, or what mediates the process of self-monitoring. Kanfer \citep{kanfer_self-monitoring:_1970} suggests that behavior change is a result of the self-awareness that one generates while self-monitoring. In contrast, Nelson and Hayes [ref] suggest that the greater self-monitoring process is what mediates change. Per Nelson and Hayes, the self-monitoring process includes ``therapist instructions, training, self-recording device, self-monitoring responses''. The focus on the greater environment surrounding self-monitoring accounts for behavior change even when self-monitoring is not consistent and detailed. This uncertainty in understanding the role of self-monitoring is partly due to limitations in the ability to evaluate the self-monitoring process. \citep{Baranowski1994, kanfer_self-monitoring:_1970, burke_self-monitoring_2005, glanz_improving_2006}. 

In the next chapter I further explore how electronic self-monitoring tools for dietary intake are evaluated. This provides insight into understanding the effectiveness of different features of self-monitoring tools. 


\subsection{Technology and Behavior Change (Persuasive Technology)}
In this section, talk about the body of work that has looked at how to use technology to support behavior change. Persuasive technology \citep{Fogg2002} and theory-driven technology \citep{consolvo_theory-driven_2009}. 

Persuasive technology is technology designed to change behavior or attitudes. Technology can play three different roles: tools, or as a means to help with a task; media, or as a means to provide content; or social actor, or like a living entity (\citep{Fogg2002}). Technology can change attitudes or behaviors by engaging these three different roles to make desired outcomes easier to achieve. Fogg identifies seven types of tools that support this (\citep{Fogg2002}.  \textit{Reduction} supports the task of change by simplifying something that is challenging or complex. \textit{Tunneling} guides the individual through the process of change. \textit{Tailoring} provides customized information for an individual, which makes it easier for the individual to understand, relate to, and take action on. \textit{Suggestion} uses context to intervene at the moment of maximum relevance. \textit{Self-monitoring} makes it easy or enjoyable to monitor their attitudes or behaviors. \textit{Surveillance} allows an individual to be monitored, supporting behavior change. \textit{Conditioning} uses principles of operant conditioning to change behavior by reinforcing the use of new behaviors. 





\section{Summary}
In this section, I reviewed relevant background information for considering how technology can support nutritional behavior change. This included a individual model of how people choose their target behavior (Theory of Planned Behavior) and how environment influences behavior (Social-Cognitive Theory). I then introduced some therapy approaches used by experts to guide the process of change. This includes Cognitive-Behavioral Theory, which teaches people how to  use cognitive strategies to problem-solve challenges they face in changing their behavior, and Motivational Interviewing, which employs an expert to help people identify their goals and monitor their progress toward them. MI depends on goal-setting theory and self-monitoring. Next, I described a theoretical construct that applies within all of these theories: the Trans-Theoretical model that describes the overall process of change. Finally, I consider the role of technology in the context of persuading and supporting behavior change. In the next chapter, I further explore technology that supports behavior change and relate it back to the theories presented in this chapter. 
