\chapter{Background and Related Work}
\label{cha:relatedWork}

The cognitive processes involved in the recall of food have been described in children as (Baranowski \& Domel, 1994):
�    Attention
�    Perception
�    Organisation
�    Retention
�    Retrieval
�    Response formulation

\section{Background}
\subsection{Self-Monitoring for Health}
Self-monitoring has been key to supporting lifestyle behavior changes necessary for treating cardiovascular disease, diabetes, cancer, and renal disease. 

Electronic self-monitoring has advantages over pencil and paper self-monitoring. 

Certain lifestyle behaviors can help to prevent disease as well as treat it. 

Since many of the behavior changes relate to nutrition and dietary intake, much research focuses on this. However, nutrition literature demonstrates that brief monitoring of the behaviors can still be effective. The technology domain has not put much emphasis on investigating brief monitoring instruments. 

Self-monitoring can help people make the behavior changes that prevent disease. 



\subsection{Evaluation of Electronic Self-Monitoring}
In this section, I'll talk about the evaluations of mobile devices for self-monitoring of dietary intake. There are a couple of ways evaluations can be grouped. First, in terms of whether the evaluation is done in terms of technology (evaluating the technology, interface, or how people use it), or in terms nutrition/medical (evaluating the support it provides for dietary change). While there is a rich literature in the medical and nutrition community about the use of electronic self-monitoring to support dietary or physical activity behavior change, much of that focuses on whether a tool in general improves the self-monitoring process. In the technology literature, evaluations focus more on documenting the design process and whether the technology works in regards to the intended impact. 

I haven't seen any work published that focuses on evaluating how different features of an electronic self-monitoring tool impact the ability of an individual to self-monitor over time. 

\subsubsection{Nutrition-Related work}

We have the nutrition community, where most self-monitoring of dietary intake is done in conjunction with a larger program, for example, a weight loss program that provides a lot of support for individuals (including tech support, personalizing software, and providing timely, personalized feedback on a regular basis). I specifically do not include or address research focused on the use of electronic self-monitoring to monitor disease conditions such as diabetes. The other concern of nutritionists is the task of collecting detailed information about a population's dietary intake to analyze it in terms of a number of variables (either descriptively, or to identify correlations between dietary factors and disease). Related research also includes work that is investigating how particular dietary changes impact disease; in this case, participants are required to self-monitor their dietary intake in order to confirm that they are following specific, prescribed nutritional goals. Researchers report on how different forms of self-monitoring support this larger research program. 

In the weight loss program research, research reporting on self-monitoring of dietary intake consists of studies where people participating in an existing weight loss/management program use electronic self-monitoring. Participants are already participating in a successful program consisting of education, group meetings on a weekly/regular basis, and regular, personalized feedback. In these studies, participants who do use electronic self-monitoring adhere to the program better/longer, and tend to lose more weight than participants who aren't self-monitoring. Research in this area characteristically reports large number of participants monitoring for a long period of time. 

In the case of gathering detailed information about dietary intake for epidemiological purposes, the goal is to collect very correct data. The concern in this case is that the longer the time between eating and documenting, the greater the chance of error in the record. Therefore, the researchers want very timely records to improve the quality of the data. However, the populations are generally less motivated to document their food intake. It's also possible that the populations are less nutritionally literate, and usually don't have extensive training in the use of the monitoring instruments. Research that fits this profile tends to collect dietary intake information on the scale of a few days, rather than weeks. This puts the participants at a disadvantage, as dietary intake instruments usually require a few days (at least) to become familiar, and personalization or customization of the instrument doesn't result in much benefit. Research in this area characteristically reports many participants tracking for a short period of time (~3-7 days).

The final group of nutrition oriented research is concerned with the impact of various dietary recommendations over a long term. The Women's Health Initiative is an example. The Women's Health Initiative is tasked with studying the incidence of disease in women who have committed to following particular dietary guidelines, specifically low fat with high fruit, vegetable and whole grain intake. Women in this study need to document what they eat to provide verification to the program that they are following the diet. Here, the primary focus of the research is to support the participating women in following the diet, therefore the research program provides substantial resources and support to the participants, and additionally, the participants are highly motivated. Electronic self-monitoring has been studied in the context of making the recording process easier. Research in this area has the characteristics of large numbers of participants self-monitoring for a long period of time (1-6 months). 

In all of the above mentioned research, the self-monitoring instruments are frequently commercially available software, and when instruments are developed for the work, they are rarely described in detail. 

\subsubsection{Technology-oriented work}
In the technology-oriented domain, research focuses more on the design and development of technologies to support new self-monitoring technologies, or architecture to support self-monitoring. I am restricting this discussion to  work related to the mobile self-monitoring of dietary intake, although sometimes the work includes related behaviors such as physical activity or overall wellness. This discussion specifically excludes work that includes only a web-based component. 

One body of work in the technology areas focuses primarily on understanding specific populations, such as individuals with diabetes or renal disease, and mobile information needs specific to these populations. This includes both characteristics of the population (for example, renal disease sufferers tend to be older, etc) and the disease or treatment of the disease (for example, renal disease impacts eyesite and requires certain dietary restrictions). While these characteristics don't always pertain to a general population not suffering from these diseases, some insights that result from the tool design and evaluation do make sense. 

Published evaluations of mobile/electronic self-monitoring of dietary intake tools vary widely. HyperFit papers report 4 different evaluations, ranging from being used by individuals, to weight loss support groups, and by nutrition coaches, either by themselves or by clients. Most evaluations are 2-4 weeks, with a notable exception of WellnessDiary, which was used for 3 months by a population of people who wanted to lose weight. 

Apart from the work that focuses on a specific population, research in the technology domain mostly includes participants who are using the self-monitoring technology on their own. The population is frequently described as ``overweight individuals who want to lose weight'', and sometimes an education component is included, but for the most part, the study only looks at the use of the self-monitoring technology. This differs from the nutrition-oriented research that frequently has a substantial education and support component. 

One of the main differences we see between evaluations published in the nutrition versus technology domains is that 

One similarity we see in all of this work is that the process to capture detailed food and exercise records is frequently reported as ``too difficult''. This is partly due to general tediousness of creating entries, but partly due to the databases. 

Database size ranges from non-existent ([] uses a triage approach, specifying whether a meal is healthy/unhealthy/unknown, while [] uses a 43-item survey to determine vegetable and whole grain intake) to very large (40,000-50,000). However, in all related work (except [sapofitness] which does not include detailed information about the evaluation or database), it is noted that participants found it difficult to find or enter food into the system. Two systems reported ``pre-populating'' the mobile database with user-specified common foods, but participants still encountered foods not in the system. 

Evaluations also vary widely as to whether participants carry their own device and install the software, versus being provided a device by the study. In the technology-oriented research, it is more common for participants to install the software on their own personal device, while in the nutrition studies devices are frequently provided. It may be more common for technology studies to provide the software for personal devices primarily because the research is more recent, and the availability of personally owned devices suitable for such software has increased substantially recently. This field may be more aware of good practices for ubiquitous computing evaluations. The nutrition literature, however, has been investigating this line of research for longer, and began when mobile devices were not as common in study populations. Additionally, for the research that is high impact, it may be important that everyone has the same software, which until recently has required very specific hardware. Also, for studies with many participants self-monitoring for a short time period, it is reasonable for the study to have a few devices for all participants to share. 

Reporting of measures varies depending on the goal and phase of the project. It would be helpful to be able to compare from study to study how the participants were using the software, to put it all in perspective. Usually an overall number is given: mean number of records per day. Sometimes rather than records, the number is the number of ``compliant'' days, or days when there are records entered at 3 different times. Usually there is a measure of adherence over time: the number of entered versus expected entries overall, as well as for the first week and the last week. 

Opportunity to systematically evaluate dietary self-tracking with food databases? 


\section{Related Work}
