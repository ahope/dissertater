\chapter{Background}
\label{cha:background}
This chapter covers background material that informs this dissertation. This primarily consists of the psychological and cognitive theories of behavior, learning and change that relate to self-monitoring. These theories inform the design of interventions to change behavior, either in general or specifically for nutrition behaviors. This background will highlight the role and importance of self-monitoring as a tool for nutrition behavior change. Included in this background is persuasive technology, which uses the aforementioned theories to define strategies and tactics of employing technology to nudge behavior. 

 This chapter will introduce the various theories and related concepts that impact the design and evaluation of the self-monitoring tools presented in this dissertation. 

\section{Introduction}

This chapter will cover related work, specifically work related to the self-monitoring of dietary intake. The process of self-monitoring dietary intake spans many fields: the medical community is concerned with any tools or processes that impacts a person's health; the nutrition and epidemiological community is concerned about how food impacts health, either individually or as a population; psychologists are concerned with the cognitive processes involved in both performing tasks and the higher level behavior change; and technology researchers are interested in how to design and build enabling technologies to support the process. The human-computer interaction researcher is concerned in spanning all of these fields to inform the design of the technology that ties it all together. 

For this dissertation, I focus on related work in nutrition and epidemiology, psychology and technology. My research goals relate to designing technology to support behavior change around nutrition behaviors, but research with the primary goal of collecting data via self-monitoring is helpful to relate to. Particularly in terms of beginning to identify what features of self-monitoring instruments have what goal, and characterizing how people use the different self-monitoring tools. 

Research distinguishes between self-monitoring of dietary intake to support behavior change versus as a data collection process. The distinction is that researchers know about the role of self-monitoring in the behavior change process in a number of domains. Research also makes a link between behaviors consistent with weight loss or weight maintenance. Weight loss and maintenance is concerned with monitoring calorie intake and expenditure, as well as a group of behaviors that positively correlate with good health outcomes. 

In this section, I first introduce and summarize some key theories of behavior change. These theories are important because they explain why self-monitoring is important. Understanding this context helps inform design choices. Next, I describe some common approaches clinicians and practitioners use to help patients or clients change their nutrition behaviors. These approaches can help inform the design of self-monitoring tools, either in terms of supporting or replicating expert practices. It also helps to give perspective to research that evaluates self-monitoring practices. Next, I describe some related work that focuses on designing and building technology to support self-monitoring of dietary intake. This is work that is most closely related to the work in presented in this dissertation. Finally, I synthesize research that includes evaluation of self-monitoring tools. This helps to put in perspective some of the issues I identify while evaluating the projects in this dissertation. 


\section{Behavior Change}
In this section, I review some relevant theories in regards to behavior change.  
 
\subsection{Enabling change: Social Cognitive Theory}
Social Cognitive Theory \citep{bandura_health_2004, bandura_self-efficacy:_1977} posits that a person�s behavior, environment and inner qualities all contribute to how a person functions. This theory has been applied to understanding how people learn, how social environments impact what people do, and how people regulate their own behavior. A key component in this theory is self-efficacy, which is summarized as a belief in one�s abilities.

Self-efficacy is traditionally measured by self-report. To develop self-efficacy measurements for a particular domain, researchers use open-ended approaches to identify common challenges and barriers to the problem. They then develop a series of statements of the form ``How confident are you that you can [achieve goal] even though [challenge]?'' with a 4-unit response scale ranging from ``Cannot do it'' to ``Highly certain can do''. An example of a statement is ``How confident are you that you can stick to a healthy eating plan after a long, tiring day at work?''

Research shows self-efficacy measures based on self-report indicate adherence to strategies to change behavior \citep{nothwehr_self-efficacy_2008}. While short-term studies cannot prove behavior change, self-efficacy measures provides valuable feedback about whether an intervention is supporting adherence to behavior change strategies, and indicate whether participants complete the study with an intention to continue.

This is an important feature for PI researchers: we are familiar with a domain and common challenges, so can build the scales easily; we usually use short-term studies to indicate long-term impact; and properly designed scales can help us to discover where a PI tool breaks down.

\subsection{Motivation for change: Transtheoretical Model}
The transtheoretical model (TTM) \citep{prochaska_transtheoretical_1997} reflects that people are in varying stages of change in relation to a given behavior. The stages include precontemplation, contemplation, preparation, action, maintenance and termination. 

\subsection{Supporting change: Goal setting and tending}
Here I talk about Locke and Latham \citep{locke_building_2002} and Consolvo and Landay \citep{consolvo_goal-setting_2009}. Goal theory. How to define goals. How to support progress toward goals. 

\subsection{Self-Monitoring and Change}

First, I'll talk about what self-monitoring of dietary intake is, and what is involved. Self-monitoring of dietary intake is the process that an individual uses to keep track of what they eat. The self-monitoring process is impacted by both the tool or instrument used for self-monitoring and the individual doing the self-monitoring. In regards to the individual, the process of self-monitoring involves cognitive processes, and the ability or willingness to continue self-monitoring involves individual goals, motivation, and resources.   The design of the tool impacts the process by clarifying (communicating? reminding?) goals as well as responding to individual motivation and resources. 


Research consistently supports that self-monitoring helps to mediate behavior change. Indeed, \citet{kanfer_self-monitoring:_1970} states ``self-observation is an initial step to self-directed behavior change''. Much research in this area falls outside the scope of this chapter, but in the domain of weight-loss, people who adhere to consistent self-monitoring of dietary intake lose more weight and have better indicators of related behaviors. Burke et al and the PREFER trial. WHI. \citep{greaves_systematic_2011}

It is unclear how self-monitoring impacts behavior change, or what mediates the process of self-monitoring. This is primarily due to limitations in the ability to evaluate the self-monitoring process. \citep{Baranowski1994, kanfer_self-monitoring:_1970, burke_self-monitoring_2005, glanz_improving_2006}. 

Later in this chapter I further explore how electronic self-monitoring tools for dietary intake are evaluated. 

\subsection{Self-Monitoring for Health}
Self-monitoring has been key to supporting lifestyle behavior changes necessary for treating cardiovascular disease, diabetes, cancer, and renal disease. 

Electronic self-monitoring has advantages over pencil and paper self-monitoring. 

Certain lifestyle behaviors can help to prevent disease as well as treat it. 

Since many of the behavior changes relate to nutrition and dietary intake, much research focuses on this. However, nutrition literature demonstrates that brief monitoring of the behaviors can still be effective. The technology domain has not put much emphasis on investigating brief monitoring instruments. 

Self-monitoring can help people make the behavior changes that prevent disease. 

The cognitive processes involved in the recall of food have been described in children as \citep{Baranowski1994}:
�    Attention
�    Perception
�    Organisation
�    Retention
�    Retrieval
�    Response formulation
People need to attend to the food that they are eating, at the time they eat it, in order to perceive it. The food information then needs to be organized and stored for retention. Then, the food information needs to be retrieved, and formulated into a response or action (e.g. creating a record). 

\subsection{Technology and Behavior Change (Persuasive Technology)}
In this section, talk about the body of work that has looked at how to use technology to support behavior change. Persuasive technology \citep{Fogg2002} and theory-driven technology \citep{consolvo_theory-driven_2009}. 

\section{Approaches to Changing Nutrition Behaviors}
Here I describe some standard approaches and theories nutritionists use to inform behaviors change to provide as background for reference later in this dissertation. It is not an exhaustive list. 

\subsection{Motivational Interviewing}
Motivational interviewing (MI) \citep{miller_motivational_2002, Miller2002} is a process practitioners use to identify what areas of change an individual is ready for. It is a technique employing goal setting, feedback mechanisms, and self-monitoring. There are three key elements: collaboration rather than confrontation; evoking ideas rather than imposing ideas; and autonomy rather than authority. In addition are four principles to follow: express empathy, support self-efficacy, roll with resistance, and ``develop discrepancy'' (or identify cognitive dissonance). 

MI has a strong record of impact for many types of behavior change \citep{Rubak2005}. It is particularly effective and well respected in the area of nutrition behavior changes \citep{abraham_taxonomy_2008, michie_effective_2009, greaves_systematic_2011}. 

\subsection{CBT Programs}
Cognitive-Behavioral Training (CBT) teaches people how to change the way they think about their nutrition behaviors to support them in the change. 

\subsection{Other techniques}
Other common themes in the research is regular group meetings, personalized feedback, nutritional education. 





\section{Summary}In this section, I reviewed relevant background information for this dissertation, including behavior change theories, traditional approaches specifically for supporting nutrition behavior change, and the important role of  self monitoring. I provided some overviews of how self-monitoring has been studied both from a technology development perspective, and how nutritionists evaluate self-monitoring of dietary intake from a utility perspective. I also provided a short overview of projects related to those discussed in this dissertation. 
