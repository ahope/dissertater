\chapter{Background and Related Work}
\label{cha:relatedWork}

This chapter will cover related work, specifically work related to the self-monitoring of dietary intake. The process of self-monitoring dietary intake spans many fields: the medical community is concerned with any tools or processes that impacts a person's health; the nutrition and epidemiological community is concerned about how food impacts health, either individually or as a population; psychologists are concerned with the cognitive processes involved in both performing tasks and the higher level behavior change; and technology researchers are interested in how to design and build enabling technologies to support the process. The human-computer interaction researcher is concerned in spanning all of these fields to inform the design of the technology that ties it all together. 

For this dissertation, I focus on related work in nutrition and epidemiology, psychology and technology. My research goals relate to designing technology to support behavior change around nutrition behaviors, but research with the primary goal of collecting data via self-monitoring is helpful to relate to. Particularly in terms of beginning to identify what features of self-monitoring instruments have what goal, and characterizing how people use the different self-monitoring tools. 

One theme I see in the research is the distinction between self-monitoring of dietary intake to support behavior change versus as a data collection process. The distinction is that researchers know about the role of self-monitoring in the behavior change process in a number of domains. Research also makes a link between behaviors consistent with weight loss or weight maintenance. Weight loss and maintenance is concerned with monitoring calorie intake and expenditure, as well as a group of behaviors that positively correlate with good health outcomes. 


First, I'll talk about what self-monitoring of dietary intake is, and what is involved. Self-monitoring of dietary intake is the process that an individual uses to keep track of what they eat. The self-monitoring process is impacted by both the tool or instrument used for self-monitoring and the individual doing the self-monitoring. In regards to the individual, the process of self-monitoring involves cognitive processes, and the ability or willingness to continue self-monitoring involves individual goals, motivation, and resources.   The design of the tool impacts the process by clarifying (communicating? reminding?) goals as well as responding to individual motivation and resources. 

Then, I'll talk about why researchers are interested in both studying how people self-monitor dietary intake, as well as why some people may want to self-monitor dietary intake. 

We have three groups of researchers who are interested in studying the use of self-monitoring of dietary intake. These are epidemiologists and nutritionists; technology and human-computer interaction researchers; and psychologist and cognitive scientists. These three groups work together, but the research reflects boundaries between them. The research published in these different areas focus on different aspects of the self-monitoring problem. 

\section{Behavior Change}

\subsection{Enabling change: Social Cognitive Theory}


\subsection{Motivation for change: Transtheoretical Model}
The transtheoretical model (TTM) \citep{prochaska_transtheoretical_1997} reflects that people are in varying stages of change in relation to a given behavior. The stages include precontemplation, contemplation, preparation, action, maintenance and termination. 

\subsection{Self-Monitoring and Change}
Research consistently supports that self-monitoring helps to mediate behavior change. Indeed, \citet{kanfer_self-monitoring:_1970} states ``self-observation is an initial step to self-directed behavior change''. Much research in this area falls outside the scope of this chapter, but in the domain of weight-loss, people who adhere to consistent self-monitoring of dietary intake lose more weight and have better indicators of related behaviors. Burke et al and the PREFER trial. WHI. 

It is unclear how self-monitoring impacts behavior change, or what mediates the process of self-monitoring. This is primarily due to limitations in the ability to evaluate the self-monitoring process. \citep{Baranowski1994, kanfer_self-monitoring:_1970, burke_self-monitoring_2005, glanz_improving_2006}. 

Later in this chapter I further explore how electronic self-monitoring tools for dietary intake are evaluated. 



\section{Changing Nutrition Behaviors}
What are some standard approaches to helping people change their nutrition behaviors? 

\subsection{Standard approaches}

\subsubsection{Motivational Interviewing}
Motivational interviewing (MI) is a process practitioners use to identify what areas of change an individual is ready for. 

\subsubsection{CBT Programs}
Cognitive-Behavioral Training (CBT) teaches people how to change the way they think about their nutrition behaviors to support them in the change. 

\subsubsection{Mix and Match Approaches?}
Other common themes in the research is regular group meetings, personalized feedback, nutritional education. 

\section{Technology for Self-Monitoring}
Paper and pencil were originally used for self-monitoring. From a user perspective, paper diaries were challenging because they required the user to remember to carry the diary, use it when they ate, and then perform any analysis manually (e.g., looking up calories or reviewing progress toward goals). More recently, the use of technology to support self-monitoring has included the use of computers, PDAs, websites and mobile phones with physical activity sensors. 

I will first talk about some of the projects and papers in each domain, and then address how they are evaluated overall. 

\subsection{Nutrition/Epi Research Domain}
The nutrition and epidemiological research areas include research to assess the process or outcomes of self-monitoring. There is less focus on the design or evaluation of the specific tool used for self-monitoring, and more focus on how that tool supports or does not support either the process or outcomes of self-monitoring. Research on the process of self-monitoring includes understanding how self-monitoring with various tools impacts the research process. More work focuses on outcomes of self-monitoring on larger goals, such as how self-monitoring impacts weight loss. 

\subsection{Technology Research Domain}
We see more work focused on the design and implementation of technology to support self-monitoring of dietary intake. This has recently been a very active research area. 

I will review 

\subsection{Evaluation of Electronic Self-Monitoring}
As addressed in \cite{klasnja_how_2011}, the evaluation of technology tools to support behavior change is difficult. This theme arose throughout the course of this dissertation work: what can we evaluate, how do we measure it, and what do we learn? In addition to evaluating this work individually, we wanted to be able to compare the results of our evaluation to previous work. This was more difficult than anticipated. Existing work evaluating self-monitoring of dietary intake is not consistent, and does not report results that can be used to compare one self-monitoring approach to another. 

Evaluation of technologies need to include both usability and utility: Usability reflects how usable the system is, while utility investigates the system's usefulness in a larger context.  

Given this challenge, this section focuses on the evaluation of self-monitoring of dietary intake tools. 
Much of literature in the nutrition community focuses on whether a tool in general improves the self-monitoring process. In the technology literature, evaluations focus more on documenting the design process and whether the technology works in regards to the intended impact. 

\subsubsection{Challenges}
There are many challenges to the evaluation of self-monitoring. 

\citep{junqing_shang_pervasive_2011} is developing the Dietary Data Recording System to both streamline the food data collection process and help evaluate self-monitoring technologies. It runs on an Android device and uses video and a laser grid to calculate food volume. 



\subsubsection{Nutrition-Related work}
In the nutrition community, most self-monitoring of dietary intake is done in conjunction with a larger program such as a weight loss program. Examples of this are in the Women's Health Initiative \citep{glanz_improving_2006} and SMART trial \citep{burke_self-monitoring_2011}. Both programs incorporated electronic self-monitoring as part of a larger program that included a high level of support for participants. This support included technical support,  personalization of the software, and providing timely, personalized feedback on a regular basis. 

I specifically do not include or address research focused on the use of electronic self-monitoring to monitor disease conditions such as diabetes. The other concern of nutritionists is the task of collecting detailed information about a population's dietary intake to analyze it in terms of a number of variables (either descriptively, or to identify correlations between dietary factors and disease). Related research also includes work that is investigating how particular dietary changes impact disease; in this case, participants are required to self-monitor their dietary intake in order to confirm that they are following specific, prescribed nutritional goals. Researchers report on how different forms of self-monitoring support this larger research program. 

In the weight loss program research, research reporting on self-monitoring of dietary intake consists of studies where people participating in an existing weight loss and management program use electronic self-monitoring.  Examples of this are in the Women's Health Initiative \citep{glanz_improving_2006} and SMART trial \citep{burke_self-monitoring_2011}. Both programs incorporated electronic self-monitoring as part of a larger program that included a high level of support for participants. This support included technical support,  personalization of the software, and providing timely, personalized feedback on a regular basis.  Participants are already participating in a successful program consisting of education, group meetings on a regular basis, and receiving personalized feedback. In these studies, participants who do use electronic self-monitoring adhere to the program longer, and tend to lose more weight than participants who are not self-monitoring. Research in this area characteristically reports large number of participants monitoring for a long period of time (One month for \cite{glanz_improving_2006}, and six months for \cite{burke_self-monitoring_2011}). 

In the case of gathering detailed information about dietary intake for epidemiological purposes, the goal is to collect very correct data. The concern in this case is that the longer the time between eating and documenting, the greater the chance of error in the record. Therefore, the researchers want very timely records to improve the quality of the data. However, the populations are generally less motivated to document their food intake. It's also possible that the populations are less nutritionally literate, and usually don't have extensive training in the use of the monitoring instruments. Research that fits this profile tends to collect dietary intake information on the scale of a few days, rather than weeks. This puts the participants at a disadvantage, as dietary intake instruments usually require a few days (at least) to become familiar, and personalization or customization of the instrument doesn't result in much benefit. Research in this area characteristically reports many participants tracking for a short period of time (~3-7 days).

The final group of nutrition oriented research is concerned with the impact of various dietary recommendations over a long term. The Women's Health Initiative is an example. The Women's Health Initiative is tasked with studying the incidence of disease in women who have committed to following particular dietary guidelines, specifically low fat with high fruit, vegetable and whole grain intake. Women in this study need to document what they eat to provide verification to the program that they are following the diet. Here, the primary focus of the research is to support the participating women in following the diet, therefore the research program provides substantial resources and support to the participants, and additionally, the participants are highly motivated. Electronic self-monitoring has been studied in the context of making the recording process easier. Research in this area has the characteristics of large numbers of participants self-monitoring for a long period of time (1-6 months). 

In all of the above mentioned research, the self-monitoring instruments are frequently commercially available software, and when instruments are developed for the work, they are rarely described in detail. 

\subsubsection{Technology-oriented work}
In the technology-oriented domain, research focuses more on the design and development of technologies to support new self-monitoring technologies, or architecture to support self-monitoring. I am restricting this discussion to  work related to the mobile self-monitoring of dietary intake, although sometimes the work includes related behaviors such as physical activity or overall wellness. This discussion specifically excludes work that includes only a web-based component. 

One body of work in the technology areas focuses primarily on understanding specific populations, such as individuals with diabetes or renal disease, and mobile information needs specific to these populations. This includes both characteristics of the population (for example, renal disease sufferers tend to be older, etc) and the disease or treatment of the disease (for example, renal disease impacts eyesite and requires certain dietary restrictions). While these characteristics don't always pertain to a general population not suffering from these diseases, some insights that result from the tool design and evaluation do make sense. 

Published evaluations of mobile/electronic self-monitoring of dietary intake tools vary widely. HyperFit papers report 4 different evaluations, ranging from being used by individuals, to weight loss support groups, and by nutrition coaches, either by themselves or by clients. Most evaluations are 2-4 weeks, with a notable exception of WellnessDiary, which was used for 3 months by a population of people who wanted to lose weight. 

Apart from the work that focuses on a specific population, research in the technology domain mostly includes participants who are using the self-monitoring technology on their own. The population is frequently described as ``overweight individuals who want to lose weight'', and sometimes an education component is included, but for the most part, the study only looks at the use of the self-monitoring technology. This differs from the nutrition-oriented research that frequently has a substantial education and support component. 

One of the main differences we see between evaluations published in the nutrition versus technology domains is that 

One similarity we see in all of this work is that the process to capture detailed food and exercise records is frequently reported as ``too difficult''. This is partly due to general tediousness of creating entries, but partly due to the databases. 

Database size ranges from non-existent ([] uses a triage approach, specifying whether a meal is healthy/unhealthy/unknown, while [] uses a 43-item survey to determine vegetable and whole grain intake) to very large (40,000-50,000). However, in all related work (except [sapofitness] which does not include detailed information about the evaluation or database), it is noted that participants found it difficult to find or enter food into the system. Two systems reported ``pre-populating'' the mobile database with user-specified common foods, but participants still encountered foods not in the system. 

Evaluations also vary widely as to whether participants carry their own device and install the software, versus being provided a device by the study. In the technology-oriented research, it is more common for participants to install the software on their own personal device, while in the nutrition studies devices are frequently provided. It may be more common for technology studies to provide the software for personal devices primarily because the research is more recent, and the availability of personally owned devices suitable for such software has increased substantially recently. This field may be more aware of good practices for ubiquitous computing evaluations. The nutrition literature, however, has been investigating this line of research for longer, and began when mobile devices were not as common in study populations. Additionally, for the research that is high impact, it may be important that everyone has the same software, which until recently has required very specific hardware. Also, for studies with many participants self-monitoring for a short time period, it is reasonable for the study to have a few devices for all participants to share. 

Reporting of measures varies depending on the goal and phase of the project. It would be helpful to be able to compare from study to study how the participants were using the software, to put it all in perspective. Usually an overall number is given: mean number of records per day. Sometimes rather than records, the number is the number of ``compliant'' days, or days when there are records entered at 3 different times. Usually there is a measure of adherence over time: the number of entered versus expected entries overall, as well as for the first week and the last week. 

Opportunity to systematically evaluate dietary self-tracking with food databases? 

\subsubsection{Usability Evaluation: Task identification (?)}

\subsection{Food Tasks: Food Diary Evaluation Methodology}
\label{sec:food_task_challenges}
\textbf{In this section, talk about different ways to present foods input to a food diary, and the concerns about different approaches. To show real food? Do a 24-hr recall? etc. In Cont2, I'll refer back to this. }
One area of focus was the food tasks. Since we want to apply the findings of this study to the real world, one approach is to use a food diary or recall of that participant. This gives the benefit of a person being familiar with the food, and it would allow us as researchers to identify how the interfaces do or do not support the real world food. Familiarity is important, particularly with the index-based approach, because if people do not know what is in a food they are more likely to get it wrong. However, people have different foods they are familiar with, particularly in terms of content. Familiarity can depend on age, religion, geography, vegetarian or not, even gender. However, if each participant used different foods, we would not be able to compare timing and correctness measures for the different interfaces. Also, some participants might eat more ``single component'' foods while other participants eat more combination or restaurant foods, which are more challenging to score. Therefore, we decided it was best for all participants to use the same food tasks. 

Since familiarity is so important, we carefully considered how to choose the food for the tasks. One considered approach was to refer to a published gold standard. [blah blah, not published, not an issue in nutrition research, etc. ] We considered using published diet plans, such as those advocated by the American Heart Association for healthy diets. The drawback to those is that they are recommended, and may not accurately reflect the actual diet of the participants. Additionally, since the recommended menus are based on the official Food Pyramid, and one chosen food index is also based on the Food Pyramid, this could influence the correctness scores. After consideration, we decided to choose the food tasks based on the food diaries collected in the BALANCE focus groups. Since the study populations are drawn from the same underlying populations, the issue of familiarity is addressed. More details are explained later. 

Another important aspect is how the food tasks are presented to participants. In the real world, one of the important challenges to ``perfect'' food journaling is the process of seeing/eating food, identifying what it is, sometimes identifying methods of preparation (fried versus broiled), identifying un-seen characteristics (low-fat versus full-fat milk, with butter), and identifying portion or serving sizes in order to correctly enter the food (or choose from a list in a db). Sometimes this process is easy, as with packaged foods or when preparing your own food; other times such as at a restaurant it is more challenging. Literacy is also an issue: with a traditional journaling approach, it is one thing to be able to identify one cup of spaghetti, but if you are not able to spell it properly, you might not be able to find it in a database. Due to this process, we considered that using real plates of food would be more life-like, and that using photographs of food were close to real. However, with the use of photographs, care must be taken to obtain them properly. Scale and lighting are two important considerations. And, with research, we found that there exist food photograph booklets specially prepared for the purpose of studying people's estimation of portion sizes. However, these booklets had the familiarity concerns outlined above. We decided that portion size estimation and literacy were not primary concerns for this study, so chose to present food tasks as a list of foods on a card. 


\section{Related Work}



\section{Background}
In this section, I review the work that provides a foundation for studying the self-monitoring of dietary intake. Work about how disease is impacted by dietary intake. How self-monitoring helps people to change their behaviors around dietary intake. How researchers evaluate technology and behavior change. 


\subsection{Self-Monitoring for Health}
Self-monitoring has been key to supporting lifestyle behavior changes necessary for treating cardiovascular disease, diabetes, cancer, and renal disease. 

Electronic self-monitoring has advantages over pencil and paper self-monitoring. 

Certain lifestyle behaviors can help to prevent disease as well as treat it. 

Since many of the behavior changes relate to nutrition and dietary intake, much research focuses on this. However, nutrition literature demonstrates that brief monitoring of the behaviors can still be effective. The technology domain has not put much emphasis on investigating brief monitoring instruments. 

Self-monitoring can help people make the behavior changes that prevent disease. 

The cognitive processes involved in the recall of food have been described in children as (Baranowski \& Domel, 1994):
�    Attention
�    Perception
�    Organisation
�    Retention
�    Retrieval
�    Response formulation

\subsection{Psychology/Behavior change}
I'm putting this on the back burner because I'm not sure how interesting it is. Basically, what goes in this section is something about behavior change, goal setting and tending, self-monitoring, persuasive technology stuff. 

Need to put into the bibliography:
\begin{enumerate*}
\item Locke \& Latham
\item Fogg
\item Bandura
\end{enumerate*}
 

\subsection{Evaluation of Electronic Self-Monitoring}






\section{Temp: drafting}

In this section, I'm just drafting paragraphs summarizing a single piece of work. They will be combined later. 

Siek et al \citep{siek_bridging_2009, siek_design_2006, siek_when_2006} explored the use of PDA-based self-monitoring of dietary intake for a specific population: people who have chronic kidney disease (CKD) and require dialysis. People with CKD are required to closely monitor their liquid and sodium intake. Siek identified characteristics of these patients that impacted the design of tools to support their self-monitoring. The important characteristics they identified include a need to support patients with poor literacy skills and poor vision. This research focused on a particular population with specific dietary needs. 

Siek et al  focus on the use of food diaries by people with renal failure. It turns out that the group they focus on tend to have a number of conditions that make traditional food tracking difficult for them. Patients with renal failure (ie, who have kidney dialysis on a regular basis) must track certain nutrients in their diet very carefully-in particular sodium, potassium and fluids. Tracking this information is particularly difficult for end-stage renal patients who have low literacy rates/levels. In addition, the disease and treatment cause physical side effects that make using traditional electronic food journaling applications difficult-it effects their eyesight (so small fonts or pointing targets are inappropriate) and fine muscle coordination (which makes it difficult to attain small targets). To address these issues, the researchers investigated the use of barcode scanners to specify what food someone ate. 

This study focused more on investigating how end-stage renal patients preferred to input food to a mobile nutrition journaling tool. With the choices of voice input or barcode scanning, over time participants preferred to use voice input, because it was quicker and easier than even picking up the package and using the scan. It is important to note that while participants did eventually prefer to use the quicker voice input, it was still necessary for someone to transcribe the recording to generate a food entry.  


Mamykina et al \citep{mamykina_examining_2011, mamykina_mahi:_2008} has explored supporting diabetes patients in managing their dietary intake. Specifically, MAHI focused on better supporting and exploring how to build tools to support people in their monitoring of a chronic disease, diabetes. Newly diagnosed diabetics usually go through a period of changing their dietary behaviors. MAHI is a mobile phone application that supports users in capturing and documenting eating episodes throughout the day. The records can then be reviewed later, encouraging reflection on the individual's behavior and choices. This can be characterized as a quick-capture with a strong emphasis on post-hoc analysis. Mamykina et al report \citep{mamykina_mahi:_2008} two specific items of interest for this dissertation: participants in particular had trouble using the Nokia N80 phones (including ``intimidat[ion] by the expensive-looking phone''), and there was a noticeable ``low level of personal interest'' in the use of the technology. Further work in this area focused more on social support and the building of community around diabetes management. 

Researchers at UCSD  have developed the Patient-Centered Assessment and Counseling Mobile Energy Balance (PmEB) \citep{tsai_usability_2007},  an application that runs on a cell phone to help people "self-monitor caloric balance in real-time".  The application on the phone allowed users to enter food intake and physical activity episodes, and displayed a visualization of current balance. They evaluated the application by comparing it to a paper diary, but the evaluation was designed to specifically investigate the benefit of reminders in tracking caloric balance. The participants were assigned to three groups: Paper (using a paper diary for tracking food intake and physical activity), One-Reminder (using the phone application, with the phone providing one daily reminder to enter food), and Three-Reminder (again using the phone application, but with 3 daily reminders to enter food and activity). 

How did they choose their prompts? That is, when were prompts sent? If they were really using EMI, there is supposedly some context that triggers a prompt (I would imagine something more substantial than time). I don't see any indication of that though� 

The authors write (p 180) ``The Paper group gave higher scores for how helpful their paper-based materials were for learning activity expenditures and food consumption.'' Does this supports the notion that the amount of involvement or amount of time/effort one puts into the intervention effect learning/behavior change? IE, it is arguably true that more involvement leads to more awareness leads to more effect. Would we consider the paper group or the mobile phone group to be "more involved"? Is there a way we could investigate this given the existing data? 

The authors calculate a compliance score for each group (separate for calorie consumption and expenditure) as the number of days where there was an entry divided by the number of days possible to have an entry. That is, the instrument was available for 30 days, times 5 participants  = 150 participant-days. If the 5 participants entered a total of 100 entries, compliance was calculated to be 100/150=0.6667. If the instruments for the different conditions were "the same", we could reasonably expect the compliance level to be the same for food intake for each condition (that is, we could assume all participants eat at least one thing per day, so therefore would enter at least one thing per day, if the instruments were the same). Since these numbers aren't the same, so we can conclude that there was some difference among the instruments for entering food intake. This assumption doesn't apply to exercise. Unless it is controlled for in some appropriate way, we can't assume that all participants exercise the same amount, or even the same frequency. 

One of the most comprehensive projects related to this dissertation work is the Wellness Diary. The Wellness Diary \citep{mattila_mobile_2008, ahtinen_user_2009, mattila_nuadu_2008, koskinen_customizable_2007, van_gils_feasibility_2001} is a simple, customizable diary software based on CBT. It provides many different wellness factors an individual can monitor over time. The wellness factors include quality of food intake, amount of physical activity, sleep, stress levels, and amount of time spent at work for the day. 

Matilla et al evaluated the use of the Wellness Diary by two groups of people [28]. The first group, the weight management group, was a self-selected group of 29 people who were overweight and owned a phone that could run the software. This group was asked to use the WD as part of a cognitive behavioral therapy (CBT)-based weight management program. The second group (for general wellness management) consisted of 17 volunteers who were participating in a work-related rehabilitation program, and had high burnout scores. This group was using WD to help them track stress-related indicators as part of a larger rehabilitation program. 

The weight management group was provided with the equipment and training necessary for them to implement the CBT-based weight management program, as well as suggestions about what kinds of measurements they should make (make food entries after meals, measure their weight every morning, etc.). At the end of three months, participants were classified as in either a "LOSER" group (n=12), meaning they lost more than 1\% of their starting weight, or the "OTHER" group (n = 13). The participants in the LOSER group lost significantly more weight than the OTHER group (a loss of 2.94 kg versus a gain of 0.193 kg) and made significantly more entries per day (5.78 versus 5.24). Participants in the LOSER group made a mean of 3.41 food and drink entries per day, while those in the OTHER group only made 3.05. This study provides evidence to support previous work [8] that indicate the longer a person continues their self-monitoring practices, the more weight they will lose (or the bigger the change). It also provides evidence that users appear willing to make five or six quick entries in a journaling application per day. 

The reported results from the general wellness management group focused more on general usage, usability and acceptance of the WD software. In this group, participants still made 5.48 entries per day, although there was a significant difference in the number of entries made on a weekday (6.14) compared to a weekend day (5.19). General feedback about the WD was positive, although participants were uncomfortable with entries related to stress assessment (stress, tiredness, irritation and tension). 

WellNavi \citep{wang_development_2006} investigated the use of taking photos on a mobile phone to capture food intake. Participants took a picture of the food before and after eating.  A stylus-ruler was placed next to the meal to identify scale. Participants completed a short questionnaire about non-pictured foods. The images were then analyzed by an RD for nutritional content. There was little difference in the nutrition content as identified in the photos when compared to the weighed food records and 24-hr recall. This indicated that the use of photos with the extra data (ruler, survey) is comparable to existing standards of identifying nutritional intake. The record-capture process was reported as taking an average of 5 minutes. Participants reported being concerned about someone else seeing their photos. 

LiveCompare. (?). I have a note about this, but can't find anything else!!
\section{Related Work}

SapoFitness \citep{silva_sapofitness:_2011} presents the design and prototype of a physical activity and dietary intake diary. It is an Android application. In addition to tracking physical activity and dietary intake, features include a personal profile, feedback on current health status, and sharing. It also included predefined reminders to help the user make good choices throughout the day. 

\subsection{Self-Monitoring via Photography}
A considerable amount of work is going into understanding and improving the use of mobile phones to take pictures of food as part of the self-monitoring process. 

* Taking pictures and reviewing them individually
* Taking pictures of food and having others review/analyze them. (either expert or crowd-sourced)
* Taking pictures and having them automatically analyzed 

\citep{aizawa_food_2010, kitamura_foodlog:_2009} is a website that allows users to upload images of their meals. The images are analyzed using computer vision approaches to categorize food into five food groups (grains, vegetable, meat/beans, fruit, milk) \citep{kitamura_image_2010}. The system is primarily for personal use, but includes some features for users to connect and build community. It includes a number of analytical features and capabilities. 

Almaghrabi et al \citep{almaghrabi_novel_2012} uses before and after photos of food to identify the food, look up its nutritional content, and determine how much was eaten. It addresses the measurement problem by registering the image with the users thumb. It is restricted to meals eaten on white plates. 

Zhu et al \citep{zhu_use_2010, bosch_integrated_2011, zhu_image_2010} developed an image-based analysis system that . 

* Using sensors on the body to recognize eating events. 



Self-monitoring, defined as the process of observing and recording target behavior, has been identified as a key component of behavior change in general \citep{kanfer_self-monitoring:_1970} and for weight loss in particular \citep{michie_effective_2009} \citep{burke_effect_2011}. Of the people who have lost weight and kept it off (as registered with the Weight Loss Registry), the key behavior that correlates highly with losing and keeping weight off is the practice of self-monitoring both food (energy) intake and physical activity (energy expenditure) [ref]. 

This led to the BALANCE project, where we combined physical activity detection with dietary intake monitoring to  encourage users to self-monitor. 

The research that indicates self-monitoring helps people to lose weight is not detailed enough to understand what aspects of self-monitoring are necessary. Additionally, this research is usually done on highly motivated participants. 

\section{Summary}In this section, I reviewed relevant background information for this dissertation, including behavior change theories, traditional approaches specifically for supporting nutrition behavior change, and the important role of  self monitoring. I provided some overviews of how self-monitoring has been studied both from a technology development perspective, and how nutritionists evaluate self-monitoring of dietary intake from a utility perspective. I also provided a short overview of projects related to those discussed in this dissertation. 
