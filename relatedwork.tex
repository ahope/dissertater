\chapter{Related Work}
\label{cha:relatedWork}

The previous chapter introduced behavioral theories, therapies informed by those theories, and some supporting tools and constructs. In this chapter, I describe some work that uses these concepts to build technology to support nutrition behavior change. In particular, I focus on tools that enable self-monitoring. Related work ranges from evaluating the impact of self-monitoring within a larger behavioral change program to evaluating the user experience (which includes both usability and utility) of advanced self-monitoring tools. 

\section{Introduction}
Persuasive technology is the use of technology to encourage and support people in changing their behavior. \textit{Kairos}, defined as providing useful information at a key moment, is a key characteristic of effective persuasive technologies. Mobile technologies are important to enabling kairos. In this chapter, I describe other projects that use mobile devices to support self-monitoring of nutrition behaviors. 


\section{HCI and Mobile Nutrition Tools}
There are four projects closely related to the work presented in this dissertation. They are related because they are related to nutrition as opposed to physical activity, incorporate a form of self-monitoring, they take a user-centered design and evaluation approach, and they have some form of final validation. Work not covered here includes user-centered approaches to designing technology to support self-monitoring of physical activity, work focused primarily on developing or designing new technology to support self-monitoring of dietary intake, or work focused primarily on the evaluation of commercially available self-monitoring tools. However, I do refer to some of these projects later when I discuss the evaluation of tools that support self-monitoring of dietary intake. 

The four main projects related to this dissertation work include: Siek et al's \citep{siek_bridging_2009, siek_design_2006, siek_when_2006} work on self-monitoring of fluids and nutrients for patients with renal failure; Mamykina et al's work on supporting newly diagnosed diabetes patients; Tsai et al's work on mobile phone software that tracks both physical activity and dietary intake; and the Wellness Diary by Mattila et al, which is mobile phone software to track a variety of wellness factors, including dietary intake, physical activity, mood, stress, and sleep. Work by Grimes et al (e.g. ) is excluded due to its focus on nutrition and eating behaviors within a community as opposed an individual. 

\subsection{Mobile Monitoring for Chronic Kidney Disease}

Siek et al \citep{siek_bridging_2009, siek_design_2006, siek_when_2006} explored the use of PDA-based self-monitoring of dietary intake for a specific population: people who have chronic kidney disease (CKD) and require dialysis on a regular basis. People with CKD tend to have a number of conditions that make traditional food tracking more difficult. Dialysis patients must track their intake of certain nutrients (particularly sodium, potassium and fluids) in their  very carefully. Monitoring this information is particularly difficult for end-stage renal patients with poor literacy skills. The disease and treatment cause physical side effects that make using traditional electronic food journaling applications difficult. The physical side effects include degraded eyesight and fine muscle coordination, which makes small fonts or pointing targets inappropriate. To address these issues, the researchers investigated the use of barcode scanners and voice input to make food input easier. Over time participants preferred to use voice input because it was quicker and easier than using the scanner. 

\subsection{Mobile Tools for Diabetes}
Mamykina et al \citep{mamykina_examining_2011, mamykina_mahi:_2008} has explored supporting diabetes patients in managing their dietary intake. Specifically, MAHI focused on better supporting and exploring how to build tools to support people in their monitoring of a chronic disease, diabetes. Newly diagnosed diabetics usually go through a period of changing their dietary behaviors. MAHI is a mobile phone application that supports users in capturing and documenting eating episodes throughout the day. The records can then be reviewed later, encouraging reflection on the individual's behavior and choices. This can be characterized as a quick-capture with a strong emphasis on post-hoc analysis. Mamykina et al report \citep{mamykina_mahi:_2008} two specific items of interest for this dissertation: participants in particular had trouble using the Nokia N80 phones (including ``intimidat[ion] by the expensive-looking phone''), and there was a noticeable ``low level of personal interest'' in the use of the technology. Further work in this area focused more on social support and the building of community around diabetes management. 

\subsection{Mobile Dietary Intake and Physical Activity Tracking}
Tsai et al developed the Patient-Centered Assessment and Counseling Mobile Energy Balance (PmEB) \citep{tsai_usability_2007},  an application that runs on a cell phone to help people self-monitor caloric balance in real-time.  The application on the phone allowed users to enter food intake and physical activity episodes, and displayed a visualization of current balance. They compared use of the application to self-monitoring with a paper diary. The comparison focused on characterizing the benefit of reminders in tracking caloric balance. Participants were assigned to one of three groups: paper (using a paper diary for tracking food intake and physical activity), one-reminder (using the phone application, with the phone providing one daily reminder to enter food), and three-reminder (again using the phone application, but with 3 daily reminders to enter food and activity). 


SapoFitness \citep{silva_sapofitness:_2011} presents the design and prototype of a physical activity and dietary intake diary. It is an Android application. In addition to tracking physical activity and dietary intake, features include a personal profile, feedback on current health status, and sharing. It also included predefined reminders to help the user make good choices throughout the day. 



\subsection{Mobile Wellness Tracking}
The Wellness Diary \citep{mattila_mobile_2008, ahtinen_user_2009, mattila_nuadu_2008, koskinen_customizable_2007, van_gils_feasibility_2001} is a  customizable diary software based on CBT. It provides many different wellness factors an individual can monitor over time. The wellness factors include quality of food intake, amount of physical activity, sleep, stress levels, and amount of time spent at work for the day. In regards to the self-monitoring of dietary intake, Mattila et al evaluated the use of the Wellness Diary by a self-selected group of people who were overweight and owned a phone that could run the software. This group was asked to use the Wellness Diary as part of a CBT-based weight management program. The group was provided with equipment and training necessary for them to implement the CBT-based weight management program. This included suggestions of what Wellness Diary measurements they should make (e.g. make food entries after meals, measure their weight every morning, etc). At the end of three months, participants were classified as in either a "LOSER" group (n=12), meaning they lost more than 1\% of their starting weight, or the "OTHER" group (n = 13). The participants in the LOSER group lost significantly more weight than the OTHER group (a loss of 2.94 kg versus a gain of 0.193 kg) and made significantly more entries per day (5.78 versus 5.24). Participants in the LOSER group made a mean of 3.41 food and drink entries per day, while those in the OTHER group only made 3.05. This study provides evidence to support previous work that indicate the longer a person continues their self-monitoring practices, the more weight they will lose (or the bigger the change). It also provides evidence that users appear willing to make five or six quick entries in a journaling application per day. 

\subsection{Synthesis}
The Siek and Mamykina work differ from the work presented in this dissertation in that they focus on a particular user population defined by disease. These user populations have well-defined constraints on dietary intake. Individuals who need to change their nutrition behaviors to treat a disease may be more motivated to use technology to support their necessary self-monitoring than individuals focused on preventing disease. The Wellness Diary work differs in that it includes a wide range of behaviors to monitor. Self-monitoring of dietary intake is very cursory: each record captures whether the eating episode is health, not healthy, or unknown. PmEB is closely related to one of the projects I present in this dissertation, BALANCE. The primary difference is the evaluation: PmEB collected feedback on reminding strategies, while BALANCE looked at how well participants were able to make food entries. Both BALANCE and PmEB report challenges using a food database as part of the self-monitoring tool, which contributed to the pattern-based approach that informed the rest of this dissertation. 

\subsection{An Aside: Streamlining Food Intake Capture Process}
Two other areas of related work that deserve a mention. One is that of using photography to capture a record of food eaten, detect what food was eaten, and calculate the nutrient values. The other is of using on-body sensors to detect eating episodes. One is attempting to automatically calculate food intake by observing the environment, while the other is attempting to automatically calculate food intake by observing the individual. 

The vision for using mobile-phone cameras for self-monitoring of dietary intake is a multi-step process. First, the user takes a picture of the food s/he intends to eat using the mobile phone. The process of taking the picture depends on which software is being used. The quality of the photograph, the lighting, the position of the camera all impact the ability of the software to automatically recognize the food. Some projects require the use of specific plates, while others use physical artifacts to determine camera angle and sizing information. Once the image is taken, computer vision strategies are used to identify the foods in the photo. Contextual information such as location, time and user may help the food identification process. When the user is done eating, s/he takes another photo of the food, showing the remaining food. This helps to identify how much food was eaten. A food database is used to look up the nutrition information and calculate calories consumed. 

\subsection{Using Photos to Capture Intake}
FoodLog \citep{aizawa_food_2010, kitamura_foodlog:_2009, de_silva_clustering_2011} is a website that allows users to upload images of their meals. The images are analyzed using computer vision approaches to categorize food into five food groups (grains, vegetable, meat/beans, fruit, milk) \citep{kitamura_image_2010}. The system is primarily for personal use, but includes some features for users to connect and build community. It includes a number of analytical features and capabilities. Almaghrabi et al \citep{almaghrabi_novel_2012} uses before and after photos of food to identify the food, look up its nutritional content, and determine how much was eaten. It addresses the measurement problem by registering the image with the users thumb. It is restricted to meals eaten on white plates. DietCam \citep{fanyu_kong_dietcam:_2011, Kong2011} is a mobile-phone-based camera software that is exploring how to support users to take multiple photos of a meal, before and after eating, to improve the food recognition problem. Zhu et al \citep{zhu_use_2010, bosch_integrated_2011, zhu_image_2010, kim_development_2010} developed an image-based analysis system that consists of mobile-phone based image capture (of food), a database of known foods and nutrition information, and is working on the vision and classification aspects of this problem. DiaWare \citep{shroff_wearable_2008} supplements the image-capture and recognition problem with contextual information from the mobile phone, such as user, location, and time. 
PlateMate \citep{noronha_platemate:_2011} addresses the food recognition part of the problem by crowd-sourcing nutritional analysis from food photographs. Like The Eatery.  DietSense \citep{reddy_image_2007} is a project that automatically collects images all day, rather than initiated by a user. Initial studies explored if food images could be captured, identified, and analyzed to detect food intake. Finally, \citep{wen_wu_fast_2009} took videos of people eating in fast food restaurants, used image recognition to identify food items and how much was eaten, and used a nutrition database to calculate how many calories each person ate. 

All of these projects focus on the technology behind supporting the process of self-monitoring of dietary intake. The goal is to make the capture of eating episodes automatic, because so many people find it challenging to continue self-monitoring via self-report on a continual basis. Current challenges to using images for self-monitoring include food identification, determining the amount eaten, and collecting an appropriate photograph for processing. Collection is important because users have shown that they are as uncomfortable noticeably taking photos of their food for notation purposes as they are using a food diary in social situations. It is becoming more socially acceptable as more people are documenting food for social media. Automating the process will also help. 

Apart from the technical challenges, it is how the automatic capturing of nutritional behaviors impact the self-monitoring process. Self-monitoring consists of both a moment-of-capture and a moment-of-reflection, while automatic capture reduces this to the moment-of-reflection. Since it is unclear how the process of self-monitoring impacts behavior change, this is a potential area of concern and future research opportunity. 

Finally, image-based self-monitoring may have the same challenges as traditional, nutrient-focused self-monitoring. It is still unclear how well image-techniques identify nutrients such as added fat (e.g. butter), salt, or sugar. It also keeps the user's attention focused on a nutrient-oriented approach to dietary intake, as opposed to the pattern-based approach. I discuss this further in Chapter \ref{cha:app_cont2_inlab_surveys}. 

Comber et al \citep{comber_supporting_2012} focused on using a mobile-phone based image capture system to track what patients in a hospital eat. This work is effective because it is used within a hospital environment where food served to patients is highly controlled, both in terms of content and serving size. 

\subsection{Automatically Sensing Dietary Intake}
\citep{Pabler2011} and \citep{amft_detection_2005} are both working on sensing eating activities with the use of a wearable microphone. They are investigating whether features of the chewing sounds can determine which food is being chewed, and how much. 

WellNavi \citep{wang_development_2006} investigated the use of taking photos on a mobile phone to capture food intake. Participants took a picture of the food before and after eating.  A stylus-ruler was placed next to the meal to identify scale. Participants completed a short questionnaire about non-pictured foods. The images were then analyzed by an RD for nutritional content. There was little difference in the nutrition content as identified in the photos when compared to the weighed food records and 24-hr recall. This indicated that the use of photos with the extra data (ruler, survey) is comparable to existing standards of identifying nutritional intake. The record-capture process was reported as taking an average of 5 minutes. Participants reported being concerned about someone else seeing their photos. 





\section{Evaluation of Self-Monitoring}
So far I have presented background information to situate approaches to self-monitoring for behavior change, and described some related work that focused on the design and development of technology to support various aspects of self-monitoring. Now, I describe some of the evaluation approaches these projects used to properly situate the work I describe later in this dissertation. Understanding the how self-monitoring tools have been evaluated allows us to compare new work to previous work. 

\subsection{Challenges for Self-Monitoring of Dietary Intake}
The primary challenge of evaluating dietary intake self-monitoring tools  is that it is difficult for researchers to know how correct and complete the created records are. Methods researchers use to validated food diaries include food frequency questionnaires, 24-hour recalls, and doubly labeled water. Food frequency questionnaires ask people to report how often they eat specific foods in a given time period. Twenty four-hour recalls use a well-defined process elicit an accurate list of foods eaten in the past twenty-four hours. The doubly labeled water protocol uses isotopes to calculate an accurate base energy expenditure. The base energy expenditure, the reported calories consumed and body weight can be used to calculate the correctness of reported food consumption. Food frequency questionnaires are helpful for identifying general trends in a population, but a food record for a given time period may not be consistent with the food frequency questionnaire. Doubly labeled water is a resource intensive approach and has the drawback that it can only identify the number of calories consumed if body weight does not change. The calorie consumed calculation does not help identify the correctness of specific food diary entries.  Generally, the 24-hour recall is considered the best validation approach for evaluating the correctness of reported food intake. 

Overall, existing methods for evaluating self-monitoring and health outcomes considers the overall process of self-monitoring. They are not able to provide a more detailed analysis of what features of the self-monitoring tool are most or least effective for a given task or goal. 

\framebox{
\citep{junqing_shang_pervasive_2011} is developing the Dietary Data Recording System to both streamline the food data collection process and help evaluate self-monitoring technologies. It runs on an Android device and uses video and a laser grid to calculate food volume. 

\citep{wenyan_jia_food_2009} uses an LED (attached to a camera) to take a picture of food to calculate serving size. 

\citep{mela_honest_1997} describes how self-reports of food intake are ``Honest but Invalid''. }





\subsection{Motivation for Evaluation}

\subsubsection{Impact of Self-Monitoring on Nutrition Behaviors}

The nutrition and epidemiological research areas include research to assess the process or outcomes of self-monitoring. There is less focus on the design or evaluation of the specific tool used for self-monitoring, and more focus on how that tool supports or does not support either the process or outcomes of self-monitoring. Research on the process of self-monitoring includes understanding how self-monitoring with various tools impacts the research process. More work focuses on outcomes of self-monitoring on larger goals, such as how self-monitoring impacts weight loss. 

The other concern of nutritionists is the task of collecting detailed information about a population's dietary intake to analyze it in terms of a number of variables (either descriptively, or to identify correlations between dietary factors and disease). Related research also includes work that is investigating how particular dietary changes impact disease; in this case, participants are required to self-monitor their dietary intake in order to confirm that they are following specific, prescribed nutritional goals. Researchers report on how different forms of self-monitoring support this larger research program. 

In the weight loss program research, research reporting on self-monitoring of dietary intake consists of studies where people participating in an existing weight loss and management program use electronic self-monitoring.  Examples of this are in the Women's Health Initiative \citep{glanz_improving_2006} and SMART trial \citep{burke_self-monitoring_2011}. Both programs incorporated electronic self-monitoring as part of a larger program that included a high level of support for participants. This support included technical support,  personalization of the software, and providing timely, personalized feedback on a regular basis.  Participants are already participating in a successful program consisting of education, group meetings on a regular basis, and receiving personalized feedback. In these studies, participants who do use electronic self-monitoring adhere to the program longer, and tend to lose more weight than participants who are not self-monitoring. Research in this area characteristically reports large number of participants monitoring for a long period of time (One month for \cite{glanz_improving_2006}, and six months for \cite{burke_self-monitoring_2011}). 

\subsubsection{The Use of Self-Monitoring to Understand Nutrition Behaviors}

In the case of gathering detailed information about dietary intake for epidemiological purposes, the goal is to collect very correct data. The concern in this case is that the longer the time between eating and documenting, the greater the chance of error in the record. Therefore, the researchers want very timely records to improve the quality of the data. However, the populations are generally less motivated to document their food intake. It's also possible that the populations are less nutritionally literate, and usually don't have extensive training in the use of the monitoring instruments. Research that fits this profile tends to collect dietary intake information on the scale of a few days, rather than weeks. This puts the participants at a disadvantage, as dietary intake instruments usually require a few days (at least) to become familiar, and personalization or customization of the instrument doesn't result in much benefit. Research in this area characteristically reports many participants tracking for a short period of time (~3-7 days).

\subsubsection{A Special Case: The Women's Health Initiative}
The final group of nutrition oriented research is concerned with the impact of various dietary recommendations over a long term. The Women's Health Initiative is an example. The Women's Health Initiative is tasked with studying the incidence of disease in women who have committed to following particular dietary guidelines, specifically low fat with high fruit, vegetable and whole grain intake. Women in this study need to document what they eat to provide verification to the program that they are following the diet. Here, the primary focus of the research is to support the participating women in following the diet, therefore the research program provides substantial resources and support to the participants, and additionally, the participants are highly motivated. Electronic self-monitoring has been studied in the context of making the recording process easier. Research in this area has the characteristics of large numbers of participants self-monitoring for a long period of time (1-6 months). 

In all of the above mentioned research, the self-monitoring instruments are frequently commercially available software, and when instruments are developed for the work, they are rarely described in detail. 


\subsection{Evaluation Characteristics and Measures}

I identified and reviewed 15 studies reporting on an evaluation of self-monitoring for dietary intake \citep{glanz_improving_2006, burke_effect_2011,fukuo_development_2009,atienza_using_2008, yon_personal_2007, acharya_using_2011, arsand_usability_2007,jarvinen_hyperfit:_2008, mattila_mobile_2008, kozakai_dietary_2006, reddy_image_2007, silva_sapofitness:_2011, tsai_usability_2007, wang_development_2006, long_effectiveness_2012}. These studies reflect evaluations performed \textit{in situ}. I reviewed what instrument was used (commercially available or research prototype), the number of people, recruitment strategy or population, duration, the size of the database, and the goal of the evaluation. A summary of this information is presented in Table [ref].  

% Table generated by Excel2LaTeX from sheet 'chap2'
\begin{sidewaystable}[htbp]
\small
  \centering
  \caption{Add caption}
    \begin{tabular}{rrrrrrrrrrrr}
    \toprule
    Reference & Platform & Number of Participants & Target Population & Duration & Results & Challenges & Column2 & Instrument & Goal/purpose & Database Information & Notes \\
    \midrule
    (Glanz et al. 2006) & Palm Pilot Vx & 33    & Women in Diet Modifcation arm of Women�s health Initiative. & 1 month & Entries made a mean of 5 days/wk; half ppts made entries 6-7 days/wk. 62\% of of days, entries were made >=3 different times. &       &       & Custom developed; db built from program materials, ~300 items.Tracked fat, servings of fruit, veggie \& grains. & Support participants in following a particular diet plan & \multicolumn{1}{c}{~300 items} & WHI is a multi-center initiative to study "dietary means of prevention"; 48K women in the Diet Modification arm, asked to eat a particular diet for 10 years; Looking to develop adherence strategies. Wanted to increase self-monitoring, reduce burden of monitoring, and increase adherence to WHI goals.  \\
    (L. E. Burke et al. 2011) & PDA (w/ \& w/o feedback) versus PR  (paper); & 210   & Overweight/obese (BMI), no medical conditions, & 6 months & Adherence: PDA+FB=90\%, PDA=80\%, PR=55\% &       & SMART trial, 6 mth outcomes &       &       & USDA db, 5000-6000 items &  of the calorie goal was recorded for the week.; Daily self-monitoring, group sessions, daily dietary goals (calories), weekly exercise goals; Adherence to dietary self-monitoring was counted if at least 50\%  \\
    (Acharya et al. 2011) & Same as above; DietmatePro on Palm PDAs & 210   & Same as above & 6 mths &       &       & SMART trial, 6 mth outcomes, secondary analysis &       &       &       &  \\
    (Arsand et al. 2007) & Mobile phones; 1 smart phone (touch screen), 1 "smart" feature phone (no touch screen) & 12 \& 20 & 12 w/ diabetes, 20 general & in-lab &       &       &       &       & Feedback about the design of the tool (interface) &       & Targeted for Type 2 diabetes; 4 �focus groups�, all the same people; No in situ eval \\
    (Bojic et al. 2009) & No in situ eval; Diabetes focus. & 32    & (ages 55-70) &       &       &       &       & Self built & Feedback about the design of the tool/interface &       &  \\
    (P. Jarvinen et al. 2008) & HyperFit, web \& mobile apps.; 4 trials overall. & 97 individual users;9 nutritionists;5 groups/39 participants; & First 2 trials: people interested in weight management; Third trial, tool for support in a weight management group; Fourth, as a tool for nutrition counseling. & Individuals: 2 weeks; Counseling: 3 wks; Groups: tasks of tracking for 2-3 days at a time. & �People liked it� & \multicolumn{1}{l}{��������� Nutritional data; ��������� Technology in infancy; ��������� Creating records in food \& exercise diary were too time consuming/challenging.} &       & Self-built & Feedback about the use of the entire system & 2500 items &  \\
    (Mattila et al. 2008) & Wellness Diary; (weight management study); 1.5 hr lecture on weight change via CBT & 29    & Overweight, had S60 phone & 3 months & LOSERS made more entries in all categories than OTHERS group; 79\% thought it would help them lose weight.64\% wanted to continue using it & Importance and input frequency of food \& drink decreased dramatically over the course of the study. People felt it important to be careful in observing at the beginning, but less so later.;Importance of recording weight increased over the study. &       & Wellness Diary/self built;Triage approach & Test ability of WD to support CBT-based weight management & none  &  \\
    (Kozakai et al. 2006) &       & 1     & Grad student volunteer & 2 mths &       & Food sometimes wasn�t in the database; Selected foods via menu or barcode scan.  &       & Self-built; included scale, blood pressure meter, diet diary & Test the tool & 607 w/o barcode, 805 w/ barcode &  \\
    (Reddy et al. 2007) & DietSense; Taking \& dealing with images of food eaten & 6     &       & 2 wks & Eh, was just image collection analyzed in terms of food/meals & Merely collected images + context, no self-monitoring or reflection &       & DietSense prototype, Nokia N80 phone, captured images + audio + context, location every 10 seconds & Evaluate potential usefulness for DietSense, collect data to inform future development. & 0     &  \\
    (Silva et al. 2011) & SapoFitness; Reflects little insight of persuasive design; Doesn�t identify details of �database testing� & �Several users� &       & �Several weeks� & �There was pretty good feedback from users.� & �Several &       & Self-developed Android-based system & �Several scenarios were experimented and the system performed very well, as expected. These experiments included communication with food database, daily food inserted, behaviour of the users, profile changes, and motivation of the users to use the 379 application.� & Not reported &  \\
     (Tsai et al. 2007) & PmEB; Food is chosen from a list of prechosen foods on the web; Paper, PmEB, PmEB+reminders & 15    & Clinically Overweight (BMI>=25) & 1 month &       &       &       & PmEB  &       & 750 items &  \\
    (Wang et al. 2006) & WellNavi.; Students kept 1 weighed food records, took photos, + 24 hr recall. & 28    & Students majoring in nutrition; �highly motivated�J & 2 days, 6 mths apart &       &       &       &       & Test the Wellnavi approach, of taking photos \& time to enter a record, versus the weighed record or recall & None, paper and pencil &  \\
    (Long et al. 2012) & MyPyramid tracker (website) compared to mobile phone food photographs. & 69    & Students & 3 days &       & While photos were helpful, they weren�t perfect, serving sizes were still an issue &       &       & Whether cell phone photos are an effective memory prompt for using MyPyramid tracker & USDA db, ~6000 items &  \\
    (Atienza et al. 2008) & Compared PDA-based assessment & 27    & Adults >=50yrs & 8 wks &       & Veggie intake in the PDA group increased more.; Ppts completed ~51\% of the assessments.; Week 1=75\%, Week 8=40\%. &       & PDA-based, 43-question assessment (triggered 2x/day) & Whether PDA intervention increases veggie \& whole grain intake; (CHART-D program) & None  &  \\
    (Yon et al. 2007) & 6-mth behavioral strategies/self-management skills weight loss program. Weekly meetings.; �First few weeks� were spent trouble-shooting use of software. Positive (personalized) feedback occurred on a weekly basis. & (71 PDA, 115 paper control) & Overweight, participating in weight control program & 6 months & Compared to previous study using similar protocol but with pencil/paper, and didn�t find a difference.Overall, those who self-monitored lost more. & 26\% of PDA users liked the �ease of use of entering food and exercise data�, 44\% reported they disliked the PDA/software (because they couldn�t find food they eat and couldn�t see the screen). Even with support, ppts still had trouble navigating the software and finding foods. &       &       & CalorieKing Diet Diary & Not reported in paper; current CalorieKing Diet Diary states 50,000 records &  \\
    Fukuo &       & PDA, proprietary w/ photos for serving sizes & 44 w/o diabetes, 16 w/ diabetes & 7 days & PDA agrees w/ recall &       &       &       &       &       &  \\
    \bottomrule
    \end{tabular}%
  \label{tab:addlabel}%
\end{sidewaystable}%



\subsubsection{Duration}
The duration of the evaluations I reviewed ranged from three days to six months. The three evaluations that had a duration of six months were concerned primarily with evaluating self-monitoring as a small part of a larger behavioral weight loss program. Evaluations that focused more on the technology ranged from a few days to a month. A notable exception is WellnessDiary, which was used for 3 months by a population of people who wanted to lose weight. 

\subsubsection{Participants and Populations}
The number of participants in an evaluation ranged from a single grad student volunteer to 210 participants of a behavioral weight-loss program. The studies that reported few (or ``several'') participants focused more on the feasibility of the technology. Studies with more participants reported on self-monitoring adherence and behavioral outcomes. Studies with a moderate number of participants (roughly 15-50) reported more details about challenges surrounding the technology and process of self-monitoring. 

About half of the evaluations targeted a population that was overweight or obese based on BMI. The other half recruited convenience populations. 

\subsubsection{Reported Challenges}
Five of the studies did not report any challenges associated with the self-monitoring tool or process. Four of these studies used commercially available software while one provided few details about the custom developed software. In the rest of the studies, reported challenges included the amount of effort required to create a food entry, databases that did not include desired foods, the helpfulness of reminders, and the lack of commitment to monitoring over time. 

\subsubsection{Databases}
Database size ranges from non-existent (Wellness Diary \citep{mattila_mobile_2008} uses a triage approach, specifying whether a meal is healthy, unhealthy or unknown, and \citep{atienza_using_2008} uses a 43-item survey to determine vegetable and whole grain intake) to very large (40,000-50,000) \citep{yon_personal_2007}. In all except one study it is noted that participants found it difficult to find or enter food into the system. SapoFitness \citep{silva_sapofitness:_2011} is the exception in that it does not provide enough detailed information about the evaluation.  Two systems reported pre-populating the mobile database with user-specified common foods \citep{tsai_usability_2007,glanz_improving_2006}, but participants still encountered foods not in the system. 

\subsubsection{Tool}
Evaluations vary widely as to whether participants carry their own device and install the software, versus being provided a device by the study. In the technology-oriented research, it is more common for participants to install the software on their own personal device, while in the nutrition studies devices are frequently provided. It may be more common for technology studies to provide the software for personal devices primarily because the research is more recent, and the availability of personally owned devices suitable for such software has increased substantially recently. This field may be more aware of good practices for ubiquitous computing evaluations. The nutrition literature, however, has been investigating this line of research for longer, and began when mobile devices were not as common in study populations. Additionally, for the research that is high impact, it may be important that everyone has the same software, which until recently has required very specific hardware. Also, for studies with many participants self-monitoring for a short time period, it is reasonable for the study to have a few devices for all participants to share. 

\subsubsection{Measures}
Reporting of measures varies depending on the goal and phase of the project. It would be helpful to be able to compare from study to study how the participants were using the software, to put it all in perspective. Usually an overall number is given: mean number of records per day. Sometimes rather than records, the number is the number of ``compliant'' days, or days when there are records entered at 3 different times. Usually there is a measure of adherence over time: the number of entered versus expected entries overall, as well as for the first week and the last week. 

\subsection{Evaluating Usability and User Experience of Food Diaries}

Food diaries are difficult to evaluate in lab settings due to the personal, situated nature of food consumption. As discussed above and in \citep{Baranowski1994}, three key steps in the process of self-monitoring of dietary intake are those of attention, perception and organization. The user must attend to the food intake, perceive the food, and observe the relevant features of the food. The steps of perceiving and organizing the food include recognizing what food it is, what extras may be included with it, and how much food is presented or consumed. In the real world, users are usually involved in the selection or preparation and consumption of the food, so have more context to be able to correctly identify and account for it. 

The goal of usability studies are to collect comparable data of users accomplishing well-defined tasks. This requires consistent, comparable, well-defined tasks for users to perform.  

Task presentation can be a textual list of well specified foods, with the food and serving sizes either exactly represented in the database or not. Exact specifications help to test a best-case scenario of finding and entering a food into the diary, while non-exact specifications reflect a more natural approach. However, this approach does not address the process of a user identifying the term to use to search the food database. An alternate approach to task presentation is to use photographs of food or real food. These presentation approaches are more like a real-world food specification task in that users need to generate the word to use to search the food database. However, food photographs have the downside of being challenging to identify serving sizes or added condiments. Overall, task presentation impacts which aspects of the food diary entry process a researcher can draw conclusions about. 

Task content is another area of concern in usability studies. Familiarity of foods vary from person to person. For some aspects of usability, food familiarity is not an issue.    A key challenge to finding a food in a food database is the nutritional literacy problem. 

One approach is to choose a few foods, specify exact amounts and look at how 

\textbf{In this section, talk about different ways to present foods input to a food diary, and the concerns about different approaches. To show real food? Do a 24-hr recall? etc. In Cont2, I'll refer back to this. }
One area of focus was the food tasks. Since we want to apply the findings of this study to the real world, one approach is to use a food diary or recall of that participant. This gives the benefit of a person being familiar with the food, and it would allow us as researchers to identify how the interfaces do or do not support the real world food. Familiarity is important, particularly with the index-based approach, because if people do not know what is in a food they are more likely to get it wrong. However, people have different foods they are familiar with, particularly in terms of content. Familiarity can depend on age, religion, geography, vegetarian or not, even gender. However, if each participant used different foods, we would not be able to compare timing and correctness measures for the different interfaces. Also, some participants might eat more ``single component'' foods while other participants eat more combination or restaurant foods, which are more challenging to score. Therefore, we decided it was best for all participants to use the same food tasks. 

Since familiarity is so important, we carefully considered how to choose the food for the tasks. One considered approach was to refer to a published gold standard. [blah blah, not published, not an issue in nutrition research, etc. ] We considered using published diet plans, such as those advocated by the American Heart Association for healthy diets. The drawback to those is that they are recommended, and may not accurately reflect the actual diet of the participants. Additionally, since the recommended menus are based on the official Food Pyramid, and one chosen food index is also based on the Food Pyramid, this could influence the correctness scores. After consideration, we decided to choose the food tasks based on the food diaries collected in the BALANCE focus groups. Since the study populations are drawn from the same underlying populations, the issue of familiarity is addressed. More details are explained later. 

Another important aspect is how the food tasks are presented to participants. In the real world, one of the important challenges to ``perfect'' food journaling is the process of seeing/eating food, identifying what it is, sometimes identifying methods of preparation (fried versus broiled), identifying un-seen characteristics (low-fat versus full-fat milk, with butter), and identifying portion or serving sizes in order to correctly enter the food (or choose from a list in a db). Sometimes this process is easy, as with packaged foods or when preparing your own food; other times such as at a restaurant it is more challenging. Literacy is also an issue: with a traditional journaling approach, it is one thing to be able to identify one cup of spaghetti, but if you are not able to spell it properly, you might not be able to find it in a database. Due to this process, we considered that using real plates of food would be more life-like, and that using photographs of food were close to real. However, with the use of photographs, care must be taken to obtain them properly. Scale and lighting are two important considerations. And, with research, we found that there exist food photograph booklets specially prepared for the purpose of studying people's estimation of portion sizes. However, these booklets had the familiarity concerns outlined above. We decided that portion size estimation and literacy were not primary concerns for this study, so chose to present food tasks as a list of foods on a card. 

As addressed in \cite{klasnja_how_2011}, the evaluation of technology tools to support behavior change is difficult. This theme arose throughout the course of this dissertation work: what can we evaluate, how do we measure it, and what do we learn? In addition to evaluating this work individually, we wanted to be able to compare the results of our evaluation to previous work. This was more difficult than anticipated. Existing work evaluating self-monitoring of dietary intake is not consistent, and does not report results that can be used to compare one self-monitoring approach to another. 

Evaluation of technologies need to include both usability and utility: Usability reflects how usable the system is, while utility investigates the system's usefulness in a larger context.  

Given this challenge, this section focuses on the evaluation of self-monitoring of dietary intake tools. 
Much of literature in the nutrition community focuses on whether a tool in general improves the self-monitoring process. In the technology literature, evaluations focus more on documenting the design process and whether the technology works in regards to the intended impact. 






\section{Summary}In this section, I reviewed relevant background information for this dissertation, including behavior change theories, traditional approaches specifically for supporting nutrition behavior change, and the important role of  self monitoring. I provided some overviews of how self-monitoring has been studied both from a technology development perspective, and how nutritionists evaluate self-monitoring of dietary intake from a utility perspective. I also provided a short overview of projects related to those discussed in this dissertation. 
